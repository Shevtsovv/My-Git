\documentclass[a4paper, 12pt]{article}
\usepackage[a4paper, top = 1.5cm, bottom = 1.5cm, left = 1cm, right = 1cm]{geometry}
\usepackage[english, russian]{babel}
\usepackage{graphicx}
\usepackage{subcaption}
\usepackage{mathtools}
\usepackage{amsfonts}
\usepackage{float}
\usepackage{float}
\title{Лабораторная работа № 4.8А "Резонанс токов"}
\author{Кирилл Шевцов Б03-402}
\date{16.10.2025}
\begin{document}
\maketitle
\section*{Лабораторная установка}
\begin{figure}[htbp]
    \centering
    \includegraphics[width=0.7\linewidth]{equip.png}
    \label{Лабораторная установка}
    \caption{Лабораторная установка}
\end{figure}
Задание предполагает снятие зависимости значений тока на учасках с амперметрами от индуктивности катушки.
Согласно установке : амперметр $A_{1}$ показывает общий ток в цепи, $A_{2}$ - ток на участке с катушкой, $A_{3}$ - ток на конденсаторе заданной
заданной емкости $C = 120$ мкФ.\newline Напряжение подается от сети постоянным $U = 220$ В, частота генератора также постоянна и равна $\nu = 50$ Гц.\newline
Картину резонанса можно увидеть либо по минимальному току на амперметре $A_{1}$, либо на осциллорафе: резонансу соответсвует нулевой сдвиг фазы,
то есть вырождение эллипса в наклонную прямую.\newline
Резонанс токов полагается исследовать на параллельном колебательном контуре, поскольку напряжение на участках цепи, параллельных включенному
вольтметру, совпадают.
\section*{Измерения и результаты}
\begin{enumerate}
    \item Будем медленно вдвигать сердечник в катушку индуктивности. Зафиксировав расстояние, на которое вдвинут сердечник, снимем показания
    амперметров $A_{1}$, $A_{2}$, $A_{3}$. Ток на учатках с катушкой, кондесатором и общий ток обозначим соответсвенно $I_{L}$, $I_{C}$, $I$.
    \begin{table}[htbp]
        \centering
        \begin{tabular}{|c|c|c|c|c|c|c|c|c|c|c|c|}
            \hline
            $x$, см & 7.0 & 6.5 & 6.0 & 5.5 & 5.0 & 4.5 & 4.0 & 3.5 & 3.0 & 2.5 \\
            $I_{L}$, А & 0.417 & 0.387 & 0.354 & 0.325 & 0.301 & 0.277 & 0.255 & 0.233 & 0.213 & 0.186 \\
            $I_{C}$, А & 0.401 & 0.395 & 0.392 & 0.393 & 0.386 & 0.398 & 0.395 & 0.395 & 0.398 & 0.391 \\
            $I$, А & 0.05 & 0.045 & 0.056 & 0.078 & 0.100 & 0.125 & 0.144 & 0.164 & 0.186 & 0.207 \\
            \hline
        \end{tabular}
        \caption{снятие токов при вдвижении сердечника}
        \label{снятие токов при вдвижении сердечника}
    \end{table}\newline
    Обозначим четкий диапазон перемещения дросселя $\Delta = 1.5 \div 6.9$ см.
    Напряжение на ЛАТР поддерживаем постоянным и равным $U_{0} = 10.0 \pm 0.1$ В.
    Частота лабораторного трансформатора $\omega = 50\pm 1$ Гц.
    Емкость конденсатора $C = 120\pm 10$ мкФ.
    \newpage
    \item Построим графики зависимостей сил тока на рассмотренных участках от положения $x$ сердечника.
    \begin{figure}[htbp]
        \centering
        \includegraphics[width=0.7\linewidth]{I_x.png}
        \label{зависимость сил тока для разных расстояний}
        \caption{зависимость силы тока от положения сердечника в катушке}
    \end{figure}\newline
    Из результатов измерений видно, что сила тока на учатке с катушкой постоянно увеличивается, общий ток в цепи уменьшается. Сила тока 
    на участке с конденсатором остается постоянной, поскольку она зависит только от частоты и напряжения генератора.
    \begin{equation}
        I_{C} = U_{0} \omega C = 2\pi \nu C U_{0}
        \label{ток на участке с конденсатором}
    \end{equation}
    На самом деле, ток меняется, как видно из графика и таблицы. Это может быть связано с тепловыми потерями и неидеальностью элементов.
    \item Найдем положение резонанса с помощью осциллографа, запишем резонансные значения тока на рассматриваемых участках цепи.
    \begin{table}[htbp]
        \centering
        \begin{tabular}{|c|c|c|c|c|c|}
            \hline
            $I^{res}_{L}$, А & $I^{res}_{C}$, А & $I^{res}$, А & $\Delta I^{res}_{L}$, А & $\Delta I^{res}_{C}$, А & $\Delta I^{res}$, А\\
            \hline
            0.428 & 0.419 & 0.049 & \multicolumn{3}{|c|}{0.001}\\
            \hline
        \end{tabular}
        \caption{резонансные токи на катушке, конденсаторе и в цепи}
        \label{резонансные токи на катушке, конденсаторе и в цепи}
    \end{table}
    \item Рассчитаем добротность колебательного контура - через токи, и резонансное сопротивление - через полный ток и напряжение.
    \begin{align}
        Q &= \frac{I^{res}_{C}}{I^{res}} = \frac{0.428}{0.049} = 8.55 \pm 0.19, \quad \Delta Q = Q\left(\frac{\Delta I^{res}_{C}}{I^{res}_{C}} + \frac{\Delta I^{res}}{I^{res}}\right) = 0.19\\
        R_{\Sigma} &= \frac{U_{0}}{I^{res}} = \frac{10.00}{0.049} = 204.08 \pm 4.37\ \text{Ом}, \quad \Delta R_{\Sigma} = R_{\Sigma}\left(\frac{\Delta U_{0}}{U_{0}} + \frac{\Delta I^{res}}{I^{res}}\right) = 4.37\ \text{Ом}
    \end{align}
    \item Рассчитаем индуктивность катушки $L_{res}$ через емкость и частоту, а затем через добротность и емкость сделаем рассчет активного
    сопротивления катушки.
    \begin{align}
        L_{res}  = \frac{1}{\omega^{2}C} = 0.084\pm 0.010\ \text{Гн}, \quad
        \Delta L_{res} = L_{res}\left(\frac{2\Delta \omega}{\omega} + \frac{\Delta C}{C}\right) = 0.010\ \text{Гн}\\
        r_L = \frac{\omega_0 L_{res}}{Q} = \frac{1}{Q} \sqrt{\frac{L_{res}}{C}} = 3.09\pm 0.49\ \text{Ом},\quad
        \Delta r_L = r_L\left(\frac{\Delta L_{res}}{2L_{res}} + \frac{\Delta C}{C} - \frac{\Delta Q}{Q}\right) = 0.49\ \text{Ом}
    \end{align}
    \newpage
    \item Сравним полученные значения сопротивления и индуктивности со значениями, снятыми с моста $E7-8$ при частоте $\nu = 50$ Гц.
    \begin{table}[htbp]
        \centering
        \begin{tabular}{|c|c|c|c|c|c}
            \hline
            Частота, Гц & \multicolumn{2}{|c|}{Табличные данные} & \multicolumn{2}{|c|}{Измеренные данные}\\
            \hline
            $\nu$, Гц & $L_{res}$, мГн & $r_{L}$, Ом &  $L_{res}$, мГн & $r_{L}$, Ом\\
            $50\pm 1$ & $67.011\pm 0.001$ & $1.937\pm 0.001$ & $84.00\pm 0.01$ & $3.09\pm 0.01$\\
            \hline
        \end{tabular}
        \caption{Сравнение с полученными данными}
        \label{Сравнение с полученными данными}
    \end{table}\\
    Как видно, измерения сопротивлений отличаются чуть меньше, чем в 2 раза, измерение индуктивности отличаются примерно на 20 мГн.
    \item Построим векторные диаграммы резонансных значений тока.
    \begin{figure}[htbp]
        \begin{subfigure}{0.45\textwidth}
            \includegraphics[width=\linewidth]{vec_diag.png}
            \caption{треугольная векторная диаграмма}
            \label{треугольная векторная диаграмма}
        \end{subfigure}
        \begin{subfigure}{0.45\textwidth}
            \includegraphics[width=\linewidth]{vec.png}
            \caption{векторная диаграмма}
            \label{векторная диаграмма}
        \end{subfigure}
        \caption{Векторная диаграмма для резонансных токов}
    \end{figure}\\
    Из графиков видно, что вектор тока на конденсаторе направлен вертикально, ток на катушке смотрит против вертикальной оси, а их сумма
    почти направлена вдоль горизонтальной оси. На первой диаграмме виден "хвостик" графика для координаты $x$ - это активное сопротивление катушки,
    которое присутствует в реальных условиях измерений. Если считать элементы идеальными, то токи $I_{L}$ и $I_{C}$ будут совпадать по величине, то есть компенсировать друг друга.\\
    \textbf{Замечание:} Таблица, согласно которой были построены данные на треугольной векторной диаграмме.
    \begin{table}[htbp]
        \centering
        \begin{tabular}{|c|c|c|}
            \hline
            Точка & $X$, см & $Y$, см\\
            $I_{C}$, начало & 0 & 0\\
            $I_{C}$, конец & 0 & 4.19\\
            $I_{L}$, начало & 0 & 0\\
            $I_{L}$, конец & 0 & -4.28\\
            $I$, начало & 0 & 0\\
            $I$, конец & 0.49 & 0\\
            \hline
            $S$, масштаб & \multicolumn{2}{|c|}{0.1}\\
            \hline
        \end{tabular}
        \caption{Таблица для векторной диаграммы}
        \label{Таблица для векторной диаграммы}
    \end{table}\\
\end{enumerate}
\end{document}