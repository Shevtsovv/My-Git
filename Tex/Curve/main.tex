\documentclass[a4paper, 12pt]{article}
\usepackage[a4paper, top = 1.5cm, bottom = 1.5cm, left = 1cm, right = 1cm]{geometry}
\usepackage{graphicx}
\usepackage{subcaption}
\usepackage{mathtools}
\usepackage{amsfonts}
\usepackage{float}
\usepackage[english, russian]{babel}
\title{Лабораторная работа "Изгиб балки"}
\author{Кирилл Шевцов, Насонов Илья, Лобанов Роман Б03-402}
% \date{8.09.2025}
\begin{document}
\maketitle
\section*{Необходимая теория}
Стержень, работающий на изгиб, называется балкой. Будем считать, что балка сделана из однородного материала, она имеет
постоянное поперечное сечение и ось симметрии.
Пусть внешние силы, приложенные к телу, также лежат в плоскости симметрии балки.
Внешние силы могут быть сосредоточенными и непрерывно распределенными, также к балке может быть приложен внешний момент сил.
\begin{figure}[htbp]
    \centering
    \includegraphics[width=0.3\linewidth]{p1.png}
    \label{устройство балки}
    \caption{Балка: 1 - ось симметрии, 2 - ось симметрии сечения, 3 - плоскость симметрии}
\end{figure}\\
Вся балка и всякая ее часть находится в равновесии. При действии на балку механического изгибающего момента сил будем
придерживаться следующего правила знаков: момент $M$ считается положительным, если он стремится изогнуть балку выпуклостью вниз, и отрицательным
в противном случае.\\
Для сил: если изгибающая сила $Q$ стремится осуществить вращение балки по часовой стрелке и отрицательной
в противном случае.
\begin{figure}[htbp]
    \centering
    \includegraphics[width=0.3\linewidth]{p2.png}
    \label{правило знаков}
    \caption{Правило знаков}
\end{figure}\\
Покажем некоторые дифференциальные соотношения для между внутренним изгибающим моментом, внутренней поперечной силой и
линейной плотностью внешней распределенной силы. Пусть по всему фрагменту балки рапределена некторая внешнаяя сила, с линейной плотностью
$\omega = \omega(z)$, ось $z$ направим вправо.
\newpage
Рассмотрим некоторый фрагмент балки $dz$
\begin{figure}[htbp]
    \centering
    \includegraphics[width=0.4\linewidth]{p3.png}
    \label{линейные силы}
    \caption{Распределения сил}
\end{figure}\\
Из условия равенства нулю равнодействующих сил и уравнения моментов относительно левого сечения
\begin{align}
    Q(z) + \omega(z)dz - Q(z + dz) &= 0 \rightarrow \frac{d}{dz}Q(z) = \omega(z)\\
    Q(z + dz)dz - M(z + dz) + M(z) - \omega dz(\frac{1}{2}dz) &= 0 \rightarrow Q = \frac{d}{dz}M(z)
    \label{моменты и силы}
\end{align}
При внимательном взгляде на геометрию изогнутой балки замечаем, что слои (волокна) балки по одну сторону от ее оси
удлиняются, а по другую сторону - укорачиваются. Слой, длина которого не изменяется, называется нейтральным. Суммарное действие касательных
напряжений дает внутреннюю поперечную силу
\begin{align}
    Q = \oint{\tau dS}
    \label{поперечная сила}
\end{align}
Нормальные напряжения при движении перпендикулярно нейтральному слою стержня, меняют знак, и момент нормальных составляющих сил
равен внутреннему изгибающему моменту
\begin{align}
    M_{x} = \oint{\sigma y dS}
    \label{изгибающий момент}
\end{align}
Здесь плечо y силы $\sigma dS$ отсчитывается от произвольной оси в плоскости сечения, параллельной нейтральному слою.
\begin{figure}[htbp]
    \centering
    \includegraphics[width=0.3\linewidth]{p4.png}
    \caption{Качественная картина распределения нормальных и касательных сил}
    \label{Качественная картина распределения нормальных и касательных сил}
\end{figure}
Можно показать, что нормальные напряжения в сечении балки не зависят от координаты по нейтральной линии сечения
и определяют изгибающий момент, равный
\begin{align}
    \sigma(x, y) = E\varepsilon = \frac{E}{r}y \rightarrow M_{x} = \oint{\sigma y dS} = \frac{E}{r}\oint{y^{2}dS} = \frac{E}{r}I_x
    \label{нормальные напряжения}
\end{align}
где вводится обозначение $I_{x} = \oint{y^{2}dS}$ - момент инерции сечения балки при повороте относительно оси $x$.
Момент инерции характеризует жесткость балки на изгиб и зависит напрямую от геометрии сечения.
\newpage
Вот примеры конкретных геометрий сечения - прямоугольное, рельс, двутавр и плоская ферма из круглых сечений
\begin{figure}[htbp]
    \centering
    \includegraphics[width=0.5\linewidth]{p5.png}
    \label{виды сечений}
    \caption{Виды сечений}
\end{figure}\\
Рассчитаем изгибающий момент в сечении балки относительно оси cимметрии оси $y$
\begin{align}
    M_{y} = \oint{\sigma x dS} = \frac{E}{r}\oint{xy dS} = \frac{E}{r}I_{xy}
    \label{момент xy}
\end{align}
где введем обозначение $I_{xy} = \oint{xy dS}$ - центробежный момент инерции сечения. В общем виде, компоненты $I_{x, y, z}, I_{xy, xz, ...}$
составляют матрицу тензора моментов инерций.\\
Выражения для оценок касательных и нормальных напряжений в удаленном сечении балки, при приложенной силе $f$, длине $l$, с характерным размером сечения $b$
\begin{align}
    \sigma \sim Pl/b^{3} \quad \tau \sim P/b^{2}
    \label{нормальные напряжения и касательные}
\end{align}
\textbf{Теорема Кастильяно}. Перемещение точки приложения сосредоточенной силы вдоль линии ее действия определяется частной
производной полной энергии упругой системы по этой силе. Зная энергию балки, и силу, на нее действующую, запишем
\begin{align}
    \delta y = \frac{\partial W}{\partial F} = \frac{1}{EI_{x}}\int_{z1}^{z2}{M \frac{\partial M}{\partial F}dz}
    \label{перемещение по Кастильяно}
\end{align}
В дальнейшем это будет полезно чтобы вычислить перемещение при изгибе балки.
Выражение для силы реакции $X$ в шарнирной опоре для балки длиной $l$
\begin{figure}[htbp]
    \centering
    \includegraphics[width=0.3\linewidth]{p6.png}
    \label{Реакция в шарнирной опоре}
    \caption{Реакция в шарнирной опоре}
\end{figure}\\
определяется из соотношений равенства нулю сил,  моментов и уравнения Кастильяно (\ref{перемещение по Кастильяно})
\begin{align}
    X = \frac{P}{2}\left(\frac{3(l/a) - 1}{(l/a)^{3}}\right)
\end{align}
Если рассмотреть $a = l/2$, то $X = 5P/16$. Справедливость этого соотношения необходимо проверить в данной работе.\\
Дифференциальное уравнение упругой линии балки
\begin{align}
    M_{x}(z) = \ddot{y}EI_{x}, \quad 1/r = \frac{\ddot{y}}{(1 + \dot{y}^{2})^{3/2}} \approx \ddot{y}
\end{align}
\newpage
\section*{Лабораторный стенд}
Лабораторная установка представлена ниже
\begin{figure}[htbp]
    \centering
    \includegraphics[width=0.4\linewidth]{p7.png}
    \label{Установка}
    \caption{Установка}
\end{figure}\\
Один конец балки прямоугольного сечения свободен, другой жестко закреплен. Вдоль балки на направляющих смонтированы два датчика силы
и два датчика перемещения. Все датчики можно переносить вдоль балки и закреплять в нужном месте. Датчики силы откалиброваны в ньютонах,
а датчики перемещения – в десятых долях миллиметра. Материал балки - сталь с модулем юнга $E = 2\cdot 10^{6}\ \text{кг/см2}$.
\section*{Выполнение работы}
\begin{enumerate}
    \item Пусть точка приложения силы находится от места жесткого закрепления балки на расстоянии $a = 30\ \text{см}$. Длина стержня $l = 67\pm 1\ \text{см}$.
    \item Толщина исследуемой балки в разных местах, ее сечение - прямоугольное.
    \begin{table}[htbp]
        \centering
        \begin{tabular}{|c|c|c|c|c|}
            \hline
            Номер измерения & 1 & 2 & 3 & 4\\
            \hline
            $l,\ \text{мм}$ & 24,4 & 24,4 & 24,35 & 24,3\\
            \hline
            $<l>,\ \text{см}$ & \multicolumn{4}{|c|}{24,36}\\
            \hline
        \end{tabular}
        \caption{Толщина балки в разных местах}
    \end{table}\\
    Положим среднюю толшину толщиной самой пластины.
    \item Проведем серию при постепенном увеличении нагрузки $P$
    \begin{table}[H]
        \centering
        \begin{tabular}{|c|c|c|}
            \hline
            Сила приложения $P$, Н & Смещение $x$, мм & Реакция $X$, Н\\
            20,3 & 0,702 & 6,5\\
            30,0 & 0,986 & 9,0\\
            40,0 & 1,338 & 12,1\\
            50,1 & 1,666 & 15,1\\
            60,1 & 1,871 & 16,9\\
            69,7 & 2,257 & 20,6\\
            80,9 & 2,780 & 25,5\\
            90,3 & 3,193 & 29,3\\
            100,2 & 3,339 & 31,6\\
            \hline
        \end{tabular}
        \caption{Толщина балки в разных местах}
    \end{table}
    \newpage
    \item Проведем серию при постепенном уменьшении нагрузки $P$
    \begin{table}[htbp]
        \centering
        \begin{tabular}{|c|c|c|}
            \hline
            Сила приложения $P$, Н & Смещение $x$, мм & Реакция $X$, Н\\
            91,5 & 3,041 & 27,9\\
            85,5 & 2,859 & 26,1\\
            80,5 & 2,716 & 24,8\\
            75,0 & 2,716 & 24,8\\
            70,6 & 2,250 & 20,3\\
            65,0 & 2,137 & 19,5\\
            60,2 & 2,058 & 18,8\\
            55,0 & 2,016 & 18,2\\
            50,2 & 1,887 & 17,0\\
            45,0 & 1,568 & 14,0\\
            40,4 & 1,543 & 13,7\\
            35,0 & 1,341 & 12,0\\
            30,2 & 1,185 & 10,4\\
            25,0 & 0,982 & 8,8\\
            20,5 & 0,679 & 6,0\\
            15,3 & 0,574 & 4,8\\
            10,2 & 0,438 & 3,7\\
            \hline
        \end{tabular}
        \caption{Толщина балки в разных местах}
    \end{table}
    \item График зависимости приложенной силы $P$ от реакции $X$\\
    **Тут картинка будет**
    \item График зависимости смещения $x$ от приложенной силы $P$\\
    **Тут картинка будет**
\end{enumerate}
\section*{Вывод}
**будет**
\end{document}
