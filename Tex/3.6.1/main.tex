\documentclass[a4paper, 12pt]{article}
\usepackage[a4paper, top = 1.5cm, bottom = 1.5cm, left = 1cm, right = 1cm]{geometry}
\usepackage[english, russian]{babel}
\usepackage{graphicx}
\usepackage{subcaption}
\usepackage{mathtools}
\usepackage{amsfonts}
\title{Лабораторная работа № 3.6.1 "Спектральный анализ электрических сигналов"}
\author{Кирилл Шевцов Б03-402}
\begin{document}
\maketitle
Задача об изучении спектров сигналов сводится к поиску отклика системы $g(t)$ на внешнее воздействие $f(t)$.
\begin{enumerate}
    \item \textbf{Исследование спектра периодических последовательностей импульсов.}\\
    Пусть на вход последовательной RLC-цепочки подается периодическая последовательностей импульсов, с длительностью $\tau$
    и периодом $T$. Изобразим спектры этого сигнала, изменяя длительность сигнала в диапазоне $50 \mu s - 200 \mu s$.
    \begin{figure}[htbp]
        \centering
        \begin{subfigure}[b]{0.4\textwidth}
            \centering
            \includegraphics[width=\textwidth]{1a.png}
            \caption{50 мкс}
            \label{fig:sub1}
        \end{subfigure}
        \begin{subfigure}[b]{0.4\textwidth}
            \centering
            \includegraphics[width=\textwidth]{1b.png}
            \caption{100 мкс}
            \label{fig:sub2}
        \end{subfigure}
        \begin{subfigure}[b]{0.4\textwidth}
            \centering
            \includegraphics[width=\textwidth]{1c.png}
            \caption{150 мкс}
        \end{subfigure}
        \begin{subfigure}[b]{0.4\textwidth}
            \centering
            \includegraphics[width=\textwidth]{1d.png}
            \caption{200 мкс}
        \end{subfigure}
        \caption{Спектры при переменном $\tau$}
        \label{fig:main}
    \end{figure}\\
    Как видно, при увеличении длительности сигнала, ширина спектра становится уже.
    Поскольку сигнал периодический, при уменьшении длительности сигнала, его ширина уменьшается $\nu_{0} \tau \sim 1$, графики это
    демонстрируют.
    \item \textbf{Исследование периодической последовательности цугов гармонических колебаний}
    Цуг - это "обрывок" синусоиды / косинусоиды. Получим спектр этого сигнала, изменяя частоту повторения импульсов.
    \begin{figure}[htbp]
        \centering
        \begin{subfigure}[b]{0.4\textwidth}
            \centering
            \includegraphics[width=\textwidth]{zug1.png}
            \caption{2 кГц}
            \label{fig:zug1}
        \end{subfigure}
        \begin{subfigure}[b]{0.4\textwidth}
            \centering
            \includegraphics[width=\textwidth]{zug2.png}
            \caption{4 кГц}
            \label{fig:zug2}
        \end{subfigure}
        \begin{subfigure}[b]{0.4\textwidth}
            \centering
            \includegraphics[width=\textwidth]{zug3.png}
            \caption{6 кГц}
            \label{fig:zug3}
        \end{subfigure}
        \begin{subfigure}[b]{0.4\textwidth}
            \centering
            \includegraphics[width=\textwidth]{zug4.png}
            \caption{8 кГц}
            \label{fig:zug4}
        \end{subfigure}
        \caption{Спектры при переменном $\tau$}
        \label{fig:zug_all}
    \end{figure}\\
    На фотографиях спектра видно, что ширина цуга неизменна и равна $\nu_{0} = 1/\tau = 20\ \text{кГц}$, частота самого большого пика соответсвует
    частоте несущей.
    \item \textbf{Исследование спектров гармонических сигналов, модулированных по амплитуде}\\
    Говорят, что сигнал модулирован по амплитуде, если
    \begin{equation}
        f(t) = a_{0}\left[1 + m\cos(\Omega t)\right]\sin(\omega t) = a(t)\sin(\omega t)
    \end{equation}
    Для амплитудно-модулированного сигнала делают приближение $\Omega \ll \omega$ (так как интересно посмотреть на амплитуды 
    близ лежащих частот), $m \ll 1$. Получим спектр амплитудно модулированного сигнала.
    \begin{align*}
        f(t) = a_{0}\sin(\omega t) + a_{0}m\cos(\Omega t)\sin(\omega t) =
        a_{0}\sin(\omega t) + \frac{a_{0}m}{2}\sin((\omega - \Omega) t) + \frac{a_{0}m}{2}\sin((\omega + \Omega) t)
    \end{align*}
    Аналогичные рассуждения, если модулированный по амплитуде сигнал содержит $\cos(\omega t)$.
    Сигнал, модулированный по амплитуде, очень сильно осциллирует (фотография сделана с прибора) поэтому фильтруют его, усредняя квадрат сигнала. Именно так
    и детектируют амплитудно-модулированный сигнал.
    \begin{align*}
        f(t) = a^{2}_{0}\left[1 + m\cos(\Omega t)\right]^{2}\sin^{2}(\omega t) = \frac{a^{2}(t)}{2}\left[1 + 2m\cos(\Omega t)\right] + \Sigma
    \end{align*}
    А затем отрезают слагаемые $\Sigma$ с высокими частотами (слагаемыми, содержащими косинусы частот $2\Omega \pm 2\omega$, $\Omega \pm 2\omega$)
    \newpage
    Сигнал, модулированный по амплитуде, с параметром глубины модуляции $m = 0.5$.
    \begin{figure}[htbp]
        \centering
        \includegraphics[width=0.5\linewidth]{am.png}
        \label{fig:am}
        \caption{Амплитудно-модулированный сигнал}
    \end{figure}\\
    Никакой полезной информации из такого сигнала без фильтрации не достать. После фильтрования сигнал выглядит понятнее.
    \begin{figure}[htbp]
        \centering
        \includegraphics[width=0.7\linewidth]{am_f.png}
        \label{fig:am_f}
        \caption{Отфильтрованный АМ - сигнал}
    \end{figure}\\
    Последний график представляет собой огибающую ам-сигнала - это и есть зашифрованная информация, которую нам пытаются донести.
    При изменении параметра глубины модуляции, график входного сигнала изменяется.
    \begin{figure}[htbp]
        \centering
        \begin{subfigure}[b]{0.3\textwidth}
            \centering
            \includegraphics[width=\textwidth]{am.png}
            \caption{m = 0.05}
            \label{fig:m1}
        \end{subfigure}
        \begin{subfigure}[b]{0.3\textwidth}
            \centering
            \includegraphics[width=\textwidth]{20m.png}
            \caption{m = 0.1}
            \label{fig:m2}
        \end{subfigure}
        \begin{subfigure}[b]{0.3\textwidth}
            \centering
            \includegraphics[width=\textwidth]{50m.png}
            \caption{m = 0.25}
            \label{fig:m3}
        \end{subfigure}
    \end{figure}
\end{enumerate}
\end{document}