\documentclass[a4paper, 12pt]{article}
\usepackage[a4paper, top = 1.5cm, bottom = 1.5cm, left = 1cm, right = 1cm]{geometry}
\usepackage[english, russian]{babel}
\usepackage{graphicx}
\usepackage{subcaption}
\usepackage{mathtools}
\usepackage{amsfonts}
\usepackage{wrapfig}
\usepackage{float}
\title{Лабораторная работа № 4.2.4 "Интерферометр Майкельсона"}
\author{Шевцов Кирилл}
\begin{document}
\maketitle
\section{Явление интерференции света}
\begin{wrapfigure}{}{0.4\textwidth}
    \includegraphics[width=\linewidth]{inter.png}
    \caption{Интерференция света}
    \label{fig:bubble_interf}
\end{wrapfigure}
Волновое уравнение для электрического поля:
\begin{align}
    E(\vec{r}, t) = E_{0}\exp\left[i(\vec{k}\vec{r} - \omega t + \varphi_{0})\right]
\end{align}
Введем величину интенсивности для волны как $I = \bar{S}$, где $\vec{S}$ -вектор Пойнтинга.\\
Рассмотрим две волны (монохроматические), имеющие одну частоту и амплитуды $a_{1}$ и $a_{2}$. Пусть в некоторой
точке они имеют фазы $\varphi_{1}$ и $\varphi_{2}$. В силу принципа суперпозиции резульвитирующее колебание в некоторой точке
будет равна сумме колебаний, которое задает каждая волна по отдельности. Если направление волны интерпретировать как вектор, длина
которого равна амплитуде этой волны, то амлитуду резульвитирующего колебания можно найти с помощью теоремы косинусов:
\begin{align}
    I = I_{1} + I_{2} + 2\sqrt{I_{1}I_{2}}\cos(\Delta \varphi)
    \label{alg:1}
\end{align}
Где введены обозначения: $I_{1} = a_{1}^{2}$, $I_{2} = a_{2}^{2}$, $\Delta \varphi = \varphi_{1} - \varphi_{2}$. Для двух одинаковых волн формула
приобретает вид:
\begin{align}
    I = 2I_{0}\left(1 + \cos(\Delta \varphi)\right)
\end{align}
Из формул видно, что если косинус угла $\Delta \varphi$ положительный, то резульвитирующая интенсивность больше суммы интенсивностей слагаемых волн,
если отрицательный - то меньше. В первом случае говорят, что слагаемые волны друг друга усиливают, во втором - ослабляют. Это явление называется явлением
интерференции.
В интерферометре волна от одного источника проходит путь до экрана по двум разным путям $z_{1}$, $z_{2}$, тогда изменение фазы:
\begin{align}
    \Delta \varphi = \frac{\omega}{c}\left(n_{2}z_{2} - n_{1}z_{1}\right) = \frac{\omega}{c}\Delta
\end{align}
Величина $\Delta$ называется оптической разностью хода. Функция косинуса принимает максимальное значение при условии:
\begin{align}
    \Delta \frac{\omega}{c} = 2\pi m \Rightarrow \Delta = cTm = \lambda m, \quad \lambda = cT = \frac{2\pi c}{\omega}
\end{align}
и минимальное при условии:
\begin{align}
    \Delta \frac{\omega}{c} = \pi \left(2m + 1\right) \Rightarrow \Delta = \frac{\lambda}{2}\left(2m + 1\right), \quad \lambda = cT = \frac{2\pi c}{\omega}
\end{align}
Отсюда получаем \textit{условие интерференционного максимума}: максимум интерференционной картины наболюдается в том случае, если оптическая разность хода равна целому числу волн.
И \textit{условие интерференционного минимума}: минимум интерференционной картины наблюдается, если оптическая разность хода равна полу-целому числу длин волн (нечетному числу длин полуволн).
\section{Интерферометр Майкельсона}
Изображен на рисунке ниже. Устройство разделяет пучок от лазера на два луча, проходящий вдоль путей и претерпевая преломление от линз и отражение от ПП. ПП расположен так, что собранные лучи собираются на экране детектора
и интерферируют, образуя картину интерференции.
\begin{figure}[htbp]
    \includegraphics[width=0.4\linewidth]{mikels.png}
    \centering
    \caption{Интерферометр Майкельсона}
    \label{fig:mikels}
\end{figure}
Если положить $n_{1} = n_{2} = 1$, то: $r_{2} - r_{1} = \text{const}$
\begin{figure}[htbp]
    \includegraphics[width=0.6\linewidth]{circl.png}
    \centering
    \caption{Интерферометр Майкельсона}
    \label{fig:circles_tol}
\end{figure}
(следует из условий минимума и максимума интерференции). 
В таком случае, если $a$ - это расстояние между изображениями источника, $L$ - расстояние между изображением $S_{1}$ и экраном, то:
\begin{align}
    \Delta (r_{n}) = r_{2} - r_{1} = \sqrt{L^{2} + r_{n}^{2}} - \sqrt{\left(L - a\right)^{2} + r_{n}^{2}} \approx a - \frac{ar_{n}^{2}}{2L\left(L - a\right)} = \text{const}
\end{align}
В центре кольца: $r = 0$, $C = \lambda m_{0} = a$ - нулевой порядок интерференции. Тогда:
\begin{align}
    a - C = \frac{ar_{n}^{2}}{2L\left(L - a\right)} = \lambda n \Rightarrow r_{n}^{2} = \frac{2nL\left(L - a\right)}{m_{0}}
    \label{alg:lambda}
\end{align}
Эта формула имеет применение в этой работе для измерения длины волны гелий - неонового лазера.
При небольшом смещении второго зеркала относительно падающего пучка на малый угол $\delta$ на экране будут видны вертикальные полосы. Причем соседние полосы являются
равноотстоящими.
\section{Экспериментальная установка}
Лабораторная установка состоит из интерферометра Майкельсона, две линзы Л1 и Л2 помогают усилить интерференционную картину. Луч света от источника преломляется через треугольную призму П,
проходит через стеклянный куб, разделяясь на два пучка, которые с помощью плеч интерференционной схемы могут при правильном совмещении попасть на экран, образуя картинку интерференции.
\begin{figure}[htbp]
    \centering
    \includegraphics[width=0.6\linewidth]{equip.png}
    \caption{Экспериментальная установка}
    \label{fig:equip}
\end{figure}\\
Установка имеет подвижное зеркало З1, соединенное с шаговым механизмом. При изменении положения зеркала З1 интерференционная картина сменяется
минимумом или максимумом. Частотометр ЧЗ-54 может работать в режиме счетчика, который будет фиксировать смену максимума и минимума картинки интерференции. Для расчетов учтем, что полное
перемещение подвижного зеркала составляет $l = 32$ мм. Зеркало З2 не может перемещаться, но может быть наклонено под некоторым углом по отношению к лучу. Для регистрации изменения интенсивности света
используется фотоэлектронный умножитель, расположенный за экраном, световой пучок попадает на ФЭУ через небольшое отверстие в экране.
\section{Выполнение работы}
\begin{enumerate}
    \item \textbf{Юстировка системы}
    \begin{enumerate}
        \item проследили, чтобы устройство гелий-неонового лазера проходил параллельно лабораторному столу на расстоянии 100 мм;
        \item Повернули кубик таким образом, чтобы лучи от лазера четко попадали в линзы, проходя пути ИЧ;
        \item Настроили положение экрана для хорошей видимости интерференции; 
    \end{enumerate}
    \item \textbf{Наблюдение интерференционной картины колец}
    Настроив интерферометр, наблюдали кольца интерференции, рисунок (\ref{fig:int}).
    \begin{figure}[htbp]
        \centering
        \includegraphics[width=0.6\linewidth]{int.jpeg}
        \caption{Кольца интерференции}
        \label{fig:int}
    \end{figure}
    \item \textbf{Измерение длины волны гелий-неонового лазера}
    \begin{enumerate}
        \item С помощью небольшой. бумаги, на которой условно выбрано начало координат, отметили радиусы последовательных колец. Кольца располагаются симметрично относительно центра, являясь
        концентрическими окружностями. Поэтому радиусы были замерены для верхних дуг окружностей интерференции.
        \begin{table}[htbp]
            \centering
            \begin{tabular}{|c|c|c|c|c|c|c|c|c|c|}
                \hline
                Порядковый номер & 1 & 2 & 3 & 4 & 5 & 6 & 7 & 8 & 9\\
                Радиус кольца, см & 0,6 & 0,9 & 1,3 & 1,5 & 1,7 & 2,0 & 2,2 & 2,3 & 2,5\\
                \hline
            \end{tabular}
        \end{table}
        \item Построим график зависимости $r_{n}(n)$
        \begin{figure}[H]
            \centering
            \includegraphics[width=0.6\linewidth]{rn.png}
            \caption{Интерференционные кольца}
        \end{figure}
        \begin{table}[htbp]
            \centering
            \begin{tabular}{|c|c|}
                \hline
                $a \pm \Delta a$ & $b \pm \Delta b$\\
                $0.751 \pm 0.022$ & $-0.601 \pm 0.124$\\
                \hline
            \end{tabular}
            \caption{Коэффиценты линейной аппроксимации $f(x) = ax + b$}
        \end{table}
    \item Рассчитаем длину волны гелий- неонового лазера, в работе присутствует линза, тогда:
    \begin{align}
        \frac{r_{n}^2}{n} &= \left[\frac{a_{1}}{a_{2}}\right]^{2}\frac{2L' (L' - a)}{m_{0}} = 0.75,\ a_{1} =15\ \text{см}\\
        \frac{1}{F} &= \frac{1}{a_{2}} + \frac{1}{a_{1}} \Rightarrow a_{2} = 6,03\ \text{см}
    \end{align}
    Расчеты дают $m_{0} = 6,6\cdot 10^{4}$ - порядок интерференции, и $\lambda_{HN} = a/m_{0} = 1207\ \text{нм}$ - длина волны. Значение больше табличного в 2 раза. Это может быть связано
    с тем, что измерить точно радиусы интерференционных колец затруднительно. Тем не менее, экспериментальные точки очень хорошо легли на теоретическую кривую.
    \end{enumerate}
    \section{Вывод}
    Явление интерференции - явление усиления или ослабления волн при их наложении. Были получены условия максимума и минимума интерференции, показана интерференционная
    картина колец, измерены их радиусы и посчитана длина волны гелий неонового лазера. Данные полученные в работе, не сошлись с табличными, поскольку было затруднительно
    поймать картину колец и измерить их радиусы.
\end{enumerate}
\end{document}