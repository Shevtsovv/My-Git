\documentclass[a4paper, 12pt]{article}
\usepackage[a4paper, top = 1.5cm, bottom = 1.5cm, left = 1cm, right = 1cm]{geometry}
\usepackage[english, russian]{babel}
\usepackage{graphicx}
\usepackage{subcaption}
\usepackage{mathtools}
\usepackage{amsfonts}
\usepackage{wrapfig}
\usepackage{multirow}
\title{Левитирующая лягушка Андре Гейма}
\author{Шевцов Кирилл Б03-402}
\begin{document}
\maketitle
\section{Общее о магнетиках}
В природе есть вещества, свойства которых проявляются при попадании их во внешнее магнитное поле. Такие вещества могут
усиливать магниное поле, либо ослаблять его, иные свойства могут проявиться при всяком изменении этого магнитного поля. Имя этих уникальных
веществ - магнетики. Магнетики разделяют на 3 основные группы: диамагнетики, парамагнетики и ферромагнетики.
\section{Модель и математический аппарат}
Для качественного описания свойства магнетиков была предложена особая модель строения вещества, а именно, что оно состоит частиц, в которых
электрон с зарядом $e$ вращается вокруг ядра. Круговое движение электрона эквивалентно круговому току $I$, который циркулирует вдоль замкнутого
контура, на который <<натянута>> поверхность площадью $S$. Для описания свойств кругового тока физики ввели величину, называемую магнитным моментом:
\begin{align}
    \vec{\mathfrak{m}} = \frac{1}{c}I\vec{S} = \frac{1}{c}IS\vec{n}
    \label{eq:m}
\end{align}
Магнитный момент аналогичен силе, которая стремится ориетировать магнитный диполь по линии внешнего магнитного поля - либо по его направлению, либо против.\newline
Для магнетиков вводится величина объемного магнитного момента - вектора намагниченности:
\begin{align}
    \vec{M} = \frac{\sum_{i = 0}^{N}\vec{\mathfrak{m}_{i}}}{V} = n\langle\vec{\mathfrak{m}}\rangle
    \label{eq:M}
\end{align}
Намагниченность показывает, насколько сильно вещество может вести себя как магнит. Про вещество с ненулевым вектором намагниченности говорят, что оно намагничено.
Парамагнетики обладают собственной намагниченностью, диамагнетики без внешнего поля - нет.\newline
Также, как и в теории электричества, в магнетизме физики ввели вектор напряженности магнитного поля $\vec{H}$. Если магнитное поле внутри магнетика равно $\vec{B}$, то:
\begin{align}
    \vec{H} = \vec{B} - 4\pi\vec{M}
    \label{eq:H}
\end{align}
Вектор напряженности введен физиками, чтобы не учитывать токи проводимости в веществе, а брать во внимание только свободные токи. Это подобно тому, как вектор электрической
индукции $\vec{D}$ зависит только от наличия свободных зарядов в проводнике.\newline
Магнитная восприимчивость и магнитная проницаемость вещества связаны соотношением:
\begin{align}
    \mu = 1 + 4\pi \chi
    \label{eq:mu}
\end{align}
Соотношение для взаимосвязи напряженности поля и магнитного поля (одно из материальных уравнений в системе уравнений Максвелла):
\begin{align}
    \vec{B} = \mu \vec{H}
    \label{eq:BmuH}
\end{align}
Это полезные характеристики для описания диамагнетиков и парамагнетиков. Для ферромагнетиков зависимости намагниченности от внешнего поля, зависимости
внешнего магнитного поля от напряженности - нелинейные.
\section{Явление диамагнетизма}
\begin{wrapfigure}{r}{0.3\textwidth}
    \includegraphics[width=\linewidth]{levi.jpg}
    \label{fig:levitation}
    \caption{\textit{Пиролитический углерод, парящий над недимовым магнитом}}
\end{wrapfigure}
Диамагнетиками являются вещества, которые не обладают магнитным моментом в отсутствии внешнего магнитного поля. Без внешнего магнитного поля на электрон в атоме
действует только сила со стороны атомного ядра и других электронов. Поэтому, если диамагнетик не помещается в магнитное поле, он не будет намагничен. Для диамагнетика с хорошей точностью
выполнятся связь между намагниченностью и напряженностью поля:
\begin{align}
    \vec{M} = \chi \vec{H}
    \label{eq:M_chi}
\end{align}
При помещении димагнетического образца в постоянное поле $B_{0}$ на движущийся электрон будет действовать сила Лоренца:
\begin{align}
    \vec{f_{B}} = -\frac{e}{c}\left[\vec{v} \times \vec{B}\right]
    \label{eq:fB}
\end{align}
Рассмотрим движение электрона в системе отсчета, которая вращается вокруг направления магнитного поля с угловой скоростью $\vec{\Omega}$, причем $\Omega \ll \omega_{e}$, где $\omega_{e}$ - угловая
скорость электрона. Во вращающейся системе отсчета к действующим на электрон силам добавятся сила Кориолиса $\vec{f_{k}}$ и центробежная сила $\vec{f_{c}}$:
\begin{align}
    \vec{f_{k}} &= 2m_{e}\left[\vec{v_{r}} \times \vec{\Omega}\right] \label{eq:fk}\\
    \vec{f_{c}} &= m_{e}\vec{\Omega} \times \left[\vec{\Omega} \times \vec{r}\right] \label{eq:fc}
\end{align}
Центробежной силой, модуль которой пропорционален $\Omega^{2}$, пренебрежем. Абсолютная скорость электрона (скорость с индексом $r$ относительная, $p$ - переносная):
\begin{align}
    \vec{v} = \vec{v_{r}} + \vec{v_{p}} = \vec{v_{r}} + \left[\vec{\Omega} \times \vec{r}\right]
    \label{eq:v_i}
\end{align}
Подставим полученную скорость в формулу Кориолиса (это допустимая замена для приближения $\Omega \ll \omega_{e}$):
\begin{align}
    \vec{f}_{1k} = \vec{f_{k}} + 2m_{e}\left[\vec{\Omega} \times \vec{r}\right] \times \vec{\Omega} = \vec{f_{k}} - \frac{1}{2}\vec{f_{c}} \approx \vec{f_{k}}
    \label{eq:f1k}
\end{align}
Теперь, угловая скорость, с которой вращается выбранная система отсчета, может быть выбирана так, чтобы уравнять силы Кориолиса и Лоренца:
\begin{align}
    2m_{e}\left[\vec{v} \times \vec{\Omega}\right] - \frac{e}{c}\left[\vec{v} \times \vec{B}\right] = \vec{0} \Rightarrow \vec{\Omega} = \frac{e}{2m_{e}c}\vec{B}
    \label{eq:Omega}
\end{align}
В такой вращающейся системе отсчета никаких новых сил, действующих на электрон нет. Частота из соотношения (\ref{eq:Omega}) называется ларморовской.
При постоянном внешнем магнитном поле движения электрона остается неизменным, но атом в целом получает дополнительное вращение с частотой $\Omega$. (\textit{Теорема Лармора}).\newline
С ларморовской частотой электрон имеет момент импульса $L = m_{e}\Omega r^2$, и обладает магнитным моментом:
\begin{align}
    \vec{\mathfrak{m}} = \frac{1}{c}IS\vec{n} = -\frac{e\pi r^{2}}{2c\pi}\vec{\Omega} = -\frac{er^{2}}{2c}\vec{\Omega} = -e^{2}\frac{\vec{B}r^{2}}{4m_{e}c^{2}}
\end{align}
Ось $Z$ перпендикулярна плоскости движения электрона, и $r^{2} = x^{2} + y^{2}$. Три направления осей координат являются равновероятными:
\begin{align}
    \langle x^{2}\rangle = \langle y^{2}\rangle = \langle z^{2}\rangle = \frac{1}{3}\langle \rho^{2}\rangle
\end{align}
Средний магнитный момент, с учетом, что атом содержит $Z$ электронов:
\begin{align}
    \langle \vec{\mathfrak{m}}\rangle = -e^{2}\frac{Z\langle \rho^{2} \rangle\vec{B}}{6m_{e}c^{2}}
\end{align}
Вектор намагниченности в таком случае:
\begin{align}
    \vec{M} = n\langle \vec{\mathfrak{m}}\rangle = -e^{2}\frac{nZ\langle \rho^{2} \rangle\vec{B}}{6m_{e}c^{2}} = \chi \vec{B}, \quad \chi < 0
\end{align}
Средний магнитный момент направлен против направления внешнего магнитного поля - это создает в диамагнетике намагниченность, также противонаправленную с магнитным полем.
Диамагнетик будет выталкиваться из области сильного магнитного поля. В этом и заключается диамагнетизм. Магнитные восприимчивости для некоторых диамагнетиков:
\begin{table}[htbp]
    \centering
    \begin{tabular}{|c|c|}
        \hline
        Вещество & Магнитная восприимчивость $\chi \times 10^{-6}$\\
        Вода (жидкая) & -13,0\\
        Нафталин & -91,8\\
        Висмут (металл) & -284,0\\
        Графит & -600\\
        \hline
    \end{tabular}
\end{table}
Самая рекордная из них - восприимчивость графита, открыта Андре Геймом. Во внешнем магнитном поле допустимого порядка человек себя
проявляет себя как диамагнетик, поскольку он на $80\%$ состоит из воды.
\section{Явление парамагнетизма}
При выводе соотношений для диамагнитного материала важно, что магнитная всприимчивость $\chi$ не зависит от величины магнитного поля и температуры образца. При наличии этих зависимостей
диамагнитные явления перекрываются так называемым парамагнитным эффектом.
\begin{figure}[htbp]
    \centering
    \begin{subfigure}[b]{0.4\textwidth}
        \includegraphics[width=\textwidth]{P1.png}
        \caption{\textit{Парамагнетик вне магнитого поля}}
        \label{fig:paramagnetic_wtB}
    \end{subfigure}
    \quad
    \begin{subfigure}[b]{0.4\textwidth}
        \includegraphics[width=\textwidth]{P2.png}
        \caption{\textit{Парамагнетик в магнитном поле}}
        \label{fig:paramagn_B}
    \end{subfigure}
\end{figure}
Парамагнетики - это вещества, которые обладают собственным магнитным моментом. Для парамагнетиков, как и диамагнетиков, выполняется соотношение (\ref{eq:M_chi}).
При отсутствии магнитного поля - все электроны в атомах
движутся беспорядочно, их магнитные моменты ориентированы хаотично. Пусть до внесения парамагнитного образца во внешнее магнитное поле электрон имеет скорость $\vec{v_{0}}$.
После внесения образца в магнитное поле электрон приобретает дополнительную скорость:
\begin{align}
    \vec{v_{i}} = \vec{v_{0}} + \vec{v_{1}} = \vec{v_{0}} + \left[\vec{\Omega} \times \vec{r}\right]
\end{align}
Тогда выражение для кинетической энергии электрона примет вид:
\begin{align}
    E = \frac{m_{e}v_{i}^{2}}{2} = \frac{m_{e}v_{0}^{2}}{2} + \left(m_{e}\vec{v_{0}}, \vec{\Omega}, \vec{r}\right) + m_{e}\Omega^{2}r^2
\end{align}
Приращение энергии электрона будет равным:
\begin{align}
    \Delta E \approx m_{e}\left(\vec{v_{0}}, \vec{\Omega}, \vec{r}\right) = \left(\vec{\Omega}, \vec{L}\right) = -\left(\vec{\mathfrak{m}}, \vec{B_{0}}\right)
\end{align}
Что соответствует потенциальной энергии магнитного диполя. Его энергия будет максимальна только в том случае, когда магнитный момент противонаправлен магнитному полю
и минимальна, если он сонаправлен с ним. В таком случае, в статистическом равновесии больше векторов магнитного
момента будет сонаправлено с вектором внешнего магнитного поля. Это и есть парамагнетизм. Вектор намагниченности имеет вид:
\begin{align}
    \vec{M} = \frac{n\mathfrak{m}^{2}}{3kT}\vec{H} = \chi\vec{H}, \quad \chi > 0
\end{align}
Восприимчивость для парамагнетика - положительная величина. Восприимчивости для некоторых парамагнетиков.
\begin{table}[htbp]
    \centering
    \begin{tabular}{|c|c|}
        \hline
        Вещество & Магнитная восприимчивость $\chi \times 10^{-5}$\\
        Вольфрам & 7,8\\
        Цезий & 6,1\\
        Аллюминий & 3,2\\
        Литий & 2,4\\
        \hline
    \end{tabular}
\end{table}
\section{Ферромагнетизм и петля гистерезиса}
Ферромагнетики - это вещества, которые могут усиливать магнитное поле, в которое они помещаются. Без отсутствия магнитного поля мангнитные моменты ориентированы
хаотично, и могут образовывать упорядоченные структуры - домены. В отличии от парамагнетиков, к силе, которая действует на атом со стороны внешнего поля, добавится
сила, действующая со стороны поля других атомов.\\
Зависимости (\ref{eq:BmuH}) и (\ref{eq:M_chi}) нелинейны в общем случае из-за кристаллической решетки ферромагнетика. Ферромагнетикам свойственна красивая петля гистерезиса.
\begin{figure}[htbp]
    \centering
    \begin{subfigure}[b]{0.25\textwidth}
        \includegraphics[width=\textwidth]{hyst.png}
        \caption{\textit{Пермаллой}}
        \label{fig:hystp}
    \end{subfigure}
    \quad
    \begin{subfigure}[b]{0.25\textwidth}
        \includegraphics[width=\textwidth]{hyst1.jpeg}
        \caption{\textit{Кремнистое железо}}
        \label{fig:hystfe}
    \end{subfigure}
    \quad
    \begin{subfigure}[b]{0.34\textwidth}
        \includegraphics[width=\textwidth]{hyst2.jpg}
        \caption{\textit{Феррит}}
        \label{fig:hystFE}
    \end{subfigure}
    \caption{\textit{Петли гистерезиса для различных веществ на экране осциллографа}}
    \label{fig:mainpic}
\end{figure}
Причем образование петли происходит в несколько этапов: намагничивание и достижения $B_{s}$ и $M_{s}$, размагничивание образца и проявление остаточной индукции, новое намагничивание
петлю гистерезиса. По петлям гистерезиса, снятым с осциллографа, можно определить величины остаточной индукции, и коэрцетивную силу - силу, необходимую для намагничивания образца ферромагнетика.
Для кривых намагничивания и зависимостей поля от напряженности, введены характеристики дифференциальной восприимчивости и проницаемости:
\begin{align}
    \mu_{\text{диф}} = \frac{dB}{dH}, \quad \chi_{\text{диф}} = \frac{dM}{dH}
\end{align}
Это означает, что при малейшем изменении поля или намагниченности коэффициенты также меняются, что ожидаемо для структуры ферромагнетика.
\begin{table}[htbp]
    \centering
    \caption{\textit{Коэрцетивные силы и индукции насыщения для некоторых ферромагнетиков}}
    \begin{tabular}{|c|c|c|}
        \hline
        Материал & Параметр & Эксперимент \\
        \hline
        \multirow{2}{*}{Пермаллой}
        & $H_c$, А/м & $9,67 \pm 1,00$ \\
        & $B_s$, Тл & $1,01 \pm 0,01$  \\
        \hline
        \multirow{2}{*}{Кремнистое железо}
        & $H_c$, А/м & $826,4 \pm 1,0$ \\
        & $B_s$, Тл & $0,20 \pm 0,01$ \\
        \hline
        \multirow{2}{*}{Феррит}
        & $H_c$, А/м & $16,3 \pm 1,0$ \\
        & $B_s$, Тл & $0,067 \pm 0,001$\\
        \hline
    \end{tabular}
\end{table}
Из таблицы ниже, кремнистое железо - жесткоупругий ферромагнетик, пермаллой и феррит - мягкоупругие ферромагнетики.
\section{Опыты с левитирующей лягушкой}
Рассмотрим длинный соленоид, у которого $N/l = n$ плотность обмотки, радиус $R$, массу лягушки обозначим $m$, лягушка находится в поле тяжести $\vec{g}$.
Лягушку в приближении будем считать диамагнитным шариком (точечным диполем) с проницаемостью $\mu$. Оценим ток $I$, который необходимо пускать по виткам
соленоида для того, чтобы лягушка левитировала. Оценим величину магнитного поля, возникающего при пропускании тока через соленоид.
\begin{enumerate}
    \item вав
\end{enumerate}
\end{document}