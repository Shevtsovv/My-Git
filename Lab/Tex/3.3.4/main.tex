\documentclass[a4paper, 12pt]{article}
\usepackage[a4paper, top = 1.5cm, bottom = 1.5cm, left = 1cm, right = 1cm]{geometry}
\usepackage[english, russian]{babel}
\usepackage{graphicx}
\usepackage{subcaption}
\usepackage{mathtools}
\usepackage{amsfonts}
\usepackage{wrapfig}
\usepackage{float}
\title{Лабораторная работа "Эффект Холла в полупроводниках"}
\author{Шевцов Кирилл Б03-402}
\begin{document}
\maketitle
\section{Эффект Холла}
\begin{wrapfigure}{r}{0.3\textwidth}
    % \centering
    \includegraphics[width=\linewidth]{Hall_effect.png}
    \label{fig:hall}
\end{wrapfigure}
Эффектом Холла называется явление, при котором на краях образца, помещенного в поперечное магнитное поле, при протекании
тока, перпендикулярного полю, возникает разность потенциалов.\\
Эффект Холла связан с природой носителей заряда тока в проводнике. Ток представляется как направленное движение множества
носителей заряда, чаще - электронов, но существуют квазичастицы - переносящие положительный заряд.\\
В электромагнетизме электроны движутся в направлении, обратном направлению тока, ток - поток положительно заряженных частиц.\\
Во внешнем магнитном поле появляется сила, действующая на заряды - сила Лоренца.
\begin{align}
    \vec{F} = \frac{1}{c}q\left[\vec{v} \times \vec{B}\right]
\end{align}
Поскольку она направлена перпендикулярно скорости заряда, то она не совершает работы, и не меняет скорость по величине.
Траектория движения заряда - окружность, радиус которой:
\begin{align}
    m_{e}\frac{v^{2}}{R_{D}} = qvB \Rightarrow R_{D} = \frac{m_{e}v}{qB}
\end{align}
Под действием силы Лоренца заряды начинают отклоняться к одной из боковых граней пластины, и скапливаться у одного из краев образца. Поэтому
возникнает электрическое поле, которое компенсирует скопление заряда у края.
\begin{align}
    q\vec{E} = \frac{q}{c}\left[\vec{v} \times \vec{B}\right] \Rightarrow qE = \frac{q}{c}vB \Rightarrow E = cvB
\end{align}
Возникающее поле образует ненулевую разность потенциалов, возникающую на краях образца.
Вектор потока $j = nq\vec{v}$, тогда $E = cjB/nq$. Величину $H = \left(nq\right)^{-1}$ называют постоянной Холла.
\section{Выполнение работы}
\begin{enumerate}
    \item Измерим калибровочную кривую электромагнита - зависимость между индукцией в зазоре электромагнита от силы тока, текущей через образец.
    \begin{table}[htbp]
        \centering
        \label{tab:1}
        \begin{tabular}{|c|c|c|c|c|c|c|c|c|c|c|c|c|}
            \hline
            $J$, мА & 0,19 & 0,36 & 0,54 & 0,72 & 0,90 & 1,08 & 1,26 & 1,44 & 1,62 & 1,80 & 1,98 & 2.01\\
            $\Phi$, мВб & 1,3 & 2,3 & 3,4 & 4,5 & 5,4 & 6,35 & 7,2 & 6,85 & 8,4 & 8,8 & 9,1 & 9,15\\
            \hline
        \end{tabular}
    \end{table}\\
    Зависимость $B(J)$ полагается нелинейной
    \begin{figure}[htbp]
        \centering
        \includegraphics[width=0.7\linewidth]{Bh.png}
        \label{fig:Bh}
    \end{figure}
    \newpage
    Из явления гистерезиса $B(H) \sim B(NJ) \sim B(J)$ - нелинейные фукнции.
    \item Измерим ЭДС Холла при разных токах $J_{r}$, текущих через образец: для этого снимем зависимость напряжение от тока $J_{M}$,текущего через электромагнит.
    \begin{table}[H]
        \centering
        \small
        \label{tab:2}
        \begin{tabular}{|c|c|c|c|c|c|c|c|c|c|c|c|}
            \hline
            $J_{r}$ & \multicolumn{11}{|c|}{0,13 мА}\\
            \hline
            $U$, мВ & 0,074 & 0,335 & 0,593 & 0,839 & 1,075 & 1,297 & 1,497 & 1,667 & 1,799 & 1,900 & 1,979\\
            $J_{M}$, мА & 0 & 0,18 & 0,36 & 0,54 & 0,72 & 0,90 & 1,08 & 1,26 & 1,42 & 1,70 & 1,88\\
            \hline
            $J_{r}$ & \multicolumn{11}{|c|}{0,3 мА}\\
            \hline
            $U$, мВ & 0,042 & 0,611 & 1,215 & 1,805 & 2,353 & 2,847 & 3,321 & 3,702 & 3,994 & 4,327 & 4,500\\
            $J_{M}$, мА & 0 & 0,18 & 0,36 & 0,54 & 0,72 & 0,90 & 1,08 & 1,26 & 1,42 & 1,70 & 1,88\\
            \hline
            $J_{r}$ & \multicolumn{11}{|c|}{0,41 мА}\\
            \hline
            $U$, мВ & 0,058 & 0,845 & 1,629 & 2,629 & 3,141 & 3,821 & 4,153 & 4,951 & 5,367 & 5,684 & 5,938\\
            $J_{M}$, мА & 0 & 0,18 & 0,36 & 0,54 & 0,72 & 0,90 & 1,08 & 1,26 & 1,44 & 1,62 & 1,81\\
            \hline
            $J_{r}$ & \multicolumn{11}{|c|}{0,52 мА}\\
            \hline
            $U$, мВ & 0,056 & 0,1025 & 2,058 & 3,091 & 3,938 & 4,788 & 5,554 & 6,199 & 6,178 & 7,106 & 7,419\\
            $J_{M}$, мА & 0 & 0,18 & 0,36 & 0,54 & 0,72 & 0,90 & 1,08 & 1,26 & 1,44 & 1,62 & 1,80\\
            \hline
            $J_{r}$ & \multicolumn{11}{|c|}{0,63 мА}\\
            \hline
            $U$, мВ & 0,027 & 1,268 & 2,471 & 3,651 & 4,747 & 5,779 & 6,716 & 7,525 & 8,153 & 8,633 & 9,010\\
            $J_{M}$, мА & 0 & 0,18 & 0,36 & 0,54 & 0,72 & 0,90 & 1,08 & 1,26 & 1,44 & 1,62 & 1,80\\
            \hline
            $J_{r}$ & \multicolumn{11}{|c|}{0,74 мА}\\
            \hline
            $U$, мВ & 0,035 & 1,395 & 2,852 & 4,295 & 5,591 & 6,814 & 7,891 & 8,879 & 9,616 & 10,147 & 10,572\\
            $J_{M}$, мА & 0 & 0,18 & 0,36 & 0,54 & 0,72 & 0,90 & 1,08 & 1,26 & 1,44 & 1,62 & 1,79\\
            \hline
            $J_{r}$ & \multicolumn{11}{|c|}{0,85 мА}\\
            \hline
            $U$, мВ & 0,033 & 1,639 & 3,257 & 4,826 & 6,309 & 7,816 & 9,037 & 10,096 & 10,972 & 11,626 & 12,062\\
            $J_{M}$, мА & 0 & 0,18 & 0,36 & 0,54 & 0,72 & 0,90 & 1,08 & 1,26 & 1,44 & 1,62 & 1,79\\
            \hline
            $J_{r}$ & \multicolumn{11}{|c|}{0,96 мА}\\
            \hline
            $U$, мВ & 0,035 & 1,821 & 3,663 & 5,353 & 7,042 & 8,595 & 9,991 & 11,177 & 12,158 & 12,893 & 13,893\\
            $J_{M}$, мА & 0 & 0,18 & 0,36 & 0,54 & 0,72 & 0,90 & 1,08 & 1,26 & 1,44 & 1,62 & 1,79\\
            \hline
        \end{tabular}
    \end{table}
    \textbf{Замечание:} При снятии напряжения с вольтметра  графики $U_{H}(B)$ стоит строить для измеренных напряжений
    за вычетом напряжения, которое измерено при нулевом токе через образец, когда в электромагнит его не вставили.\\
    (график для эдс Холла)
    \item Определив коэффициенты $k = dU_{H}/dB$, построим график $k(J_{r})$
    % Построим графики зависимости $U_{\perp}(B)$
    \item Определим знак носителей для германия: для этого (алгоритм)
    \item Рассчитаем концентрацию $n$ носителей тока, удельное сопротивление $\rho_{0}$, удельную проводимость $\sigma_{0}$, подвижность
    $\mu$ носителей.
\end{enumerate}
\end{document}