\documentclass[a4paper, 12pt]{article}
\usepackage[a4paper, top = 1.5cm, bottom = 1.5cm, left = 1cm, right = 1cm]{geometry}
\usepackage[english, russian]{babel}
\usepackage{graphicx}
\usepackage{subcaption}
\usepackage{mathtools}
\usepackage{amsfonts}
\usepackage{wrapfig}
\title{Лабораторная работа № 4.2.4 "Интерферометр Майкельсона"}
\author{Шевцов Кирилл}
\begin{document}
\maketitle
\section{Явление интерференции света}
\begin{wrapfigure}{}{0.4\textwidth}
    \includegraphics[width=\linewidth]{inter.png}
    \caption{Интерференция света}
    \label{fig:bubble_interf}
\end{wrapfigure}
Рассмотрим две волны (монохроматические), имеющие одну частоту и амплитуды $a_{1}$ и $a_{2}$. Пусть в некоторой
точке они имеют фазы $\varphi_{1}$ и $\varphi_{2}$. В силу принципа суперпозиции резульвитирующее колебание в некоторой точке
будет равна сумме колебаний, которое задает каждая волна по отдельности. Если направление волны интерпретировать как вектор, длина
которого равна амплитуде этой волны, то амлитуду резульвитирующего колебания можно найти с помощью теоремы косинусов:
\begin{align}
    I = I_{1} + I_{2} + 2\sqrt{I_{1}I_{2}}\cos(\Delta \varphi)
    \label{alg:1}
\end{align}
Где введены обозначения: $I_{1} = a_{1}^{2}$, $I_{2} = a_{2}^{2}$, $\Delta \varphi = \varphi_{1} - \varphi_{2}$. Для двух одинаковых волн формула
приобретает вид:
\begin{align}
    I = 2I_{0}\left(1 + \cos(\Delta \varphi)\right)
\end{align}
Из формул видно, что если косинус угла $\Delta \varphi$ положительный, то резульвитирующая интенсивность больше суммы интенсивностей слагаемых волн,
если отрицательный - то меньше. В первом случае говорят, что слагаемые волны друг друга усиливают, во втором - ослабляют. Это явление и называется явлением
интерференции.
В интерферометре волна от одного источника проходит путь до экрана по двум разным путям $z_{1}$, $z_{2}$, тогда изменение фазы:
\begin{align}
    \Delta \varphi = \frac{\omega}{c}\left(n_{2}z_{2} - n_{1}z_{1}\right) = \frac{\omega}{c}\Delta
\end{align}
Величина $\Delta$ называется оптической разностью хода. Функция косинуса принимает максимальное значение при условии:
\begin{align}
    \Delta \frac{\omega}{c} = 2\pi m \Rightarrow \Delta = cTm = \lambda m, \quad \lambda = cT = \frac{2\pi c}{\omega}
\end{align}
и минимальное при условии:
\begin{align}
    \Delta \frac{\omega}{c} = \pi \left(2m + 1\right) \Rightarrow \Delta = \frac{\lambda}{2}\left(2m + 1\right), \quad \lambda = cT = \frac{2\pi c}{\omega}
\end{align}
Отсюда получаем \textit{условие интерференционного максимума}: максимум интерференционной картины наболюдается в том случае, если оптическая разность хода равна целому числу волн.
И \textit{условие интерференционного минимума}: минимум интерференционной картины наблюдается, если оптическая разность хода равна полу-целому числу длин волн (нечетному числу длин полуволн).
\section{Интерферометр Майкельсона}
\begin{wrapfigure}{}{0.4\textwidth}
    \includegraphics[width=\linewidth]{mikels.png}
    \caption{Интерференция света}
    \label{fig:mikels}
\end{wrapfigure}
Интерферометр Майкельсона - это устройство, состоящее из системы двух линз и двух зеркал: волна, излучаемая
источником, разделяется на два раздельных пучка с помощью полупрозрачного зеркала, и с помощью системы двух зеркал, расположенных на пути преломленного и отраженного луча, а затем собранные с помощью полупрозрачного зеркала 
пучки попадают на экран - детектор, который помогает регистрировать картину интерференции. Линзами, стоящими на пути хода пучков, можно сделать картину интерференции более четкой. Если положить $n_{1} = n_{2} = 1$, то: $r_{2} - r_{1} = \text{const}$
\begin{figure}[htbp]
    \includegraphics[width=0.6\linewidth]{circl.png}
    \centering
    \caption{Интерферометр Майкельсона}
    \label{fig:circles_tol}
\end{figure}
(следует из условий минимума и максимума интерференции). 
В таком случае, если $a$ - это расстояние между изображениями источника, $L$ - расстояние между изображением $S_{1}$ и экраном, то:
\begin{align}
    \Delta (r) = r_{2} - r_{1} = \sqrt{L^{2} + r_{n}^{2}} + \sqrt{\left(L - a\right)^{2} + r_{n}^{2}} \approx a - \frac{ar_{n}^{2}}{2L\left(L - a\right)} = \text{const}
\end{align} 
В центре кольца: $r = 0$, $C = \lambda m_{0} = a$ - нулевой порядок интерференции. Тогда:
\begin{align}
    a - C = \frac{ar_{n}^{2}}{2L\left(L - a\right)} = \lambda n \Rightarrow r_{n}^{2} = \frac{2nL\left(L - a\right)}{m_{0}}
\end{align}
Эта формула имеет применение в этой работе для измерения длины волны гелий - неонового лазера.
При небольшом смещении второго зеркала относительно падающего пучка на малый угол $\delta$ на экране будут видны вертикальные полосы. Причем соседние полосы являются
равноотстоящими.
\section{Экспериментальная установка}
Лабораторная установка состоит из интерферометра Майкельсона, две линзы Л1 и Л2 помогают усилить интерференционную картину. Луч света от источника преломляется через треугольную призму П,
проходит через стеклянный куб, разделяясь на два пучка, которые с помощью плеч интерференционной схемы могут при правильном совмещении попасть на экран, образуя картинку интерференции.
\begin{figure}[htbp]
    \centering
    \includegraphics[width=0.6\linewidth]{equip.png}
    \caption{Экспериментальная установка}
    \label{fig:equip}
\end{figure}\\
Установка имеет подвижное зеркало З1, соединенное с шаговым механизмом. При изменении положения зеркала З1 интерференционная картина сменяется
минимумом или максимумом. Частотометр ЧЗ-54 может работать в режиме счетчика, который будет фиксировать смену максимума и минимума картинки интерференции.  Для расчетов учтем, что полное
перемещение подвижного зеркала составляет $l = 32$ мм. Зеркало З2 не может перемещаться, но может быть наклонено под некоторым углом по отношению к лучу. Для регистрации изменения интенсивности света
используется фотоэлектронный умножитель, расположенный за экраном, световой пучок попадает на ФЭУ через небольшое отверстие в экране.
\end{document}