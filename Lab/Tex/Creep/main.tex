\documentclass[a4paper, 12pt]{article}
\usepackage[a4paper, top = 1.5cm, bottom = 1.5cm, left = 1cm, right = 1cm]{geometry}
\usepackage[english, russian]{babel}
\usepackage{graphicx}
\usepackage{subcaption}
\usepackage{mathtools}
\usepackage{amsfonts}
\usepackage{float}
\title{Лабораторная работа "Ползучесть материалов"}
\author{Кирилл Шевцов, Лобанов Роман, Насонов Илья Б03-402}
\begin{document}
\maketitle
\section{Теоретические сведения}
Свойство материала деформироваться во времени при действии постоянного напряжения называется ползучестью.
Явление ползучести присуще таким материалам, как бетон, полимеры, груды, льды и металлы. Результаты ползучести ползучести
при одноосном растяжении (или сжатии) получают в виде особой кривой - кривой ползучести.
\begin{figure}[htbp]
    \centering
    \includegraphics[width=0.3\linewidth]{creep_curve.png}
    \label{fig:1}
\end{figure}
Если увеличение деформации ползучести прямо пропорционально увеличению напряжения, то ползучесть считается линейной.
\begin{figure}[htbp]
    \centering
    \includegraphics[width=0.4\linewidth]{fig2.png}
    \label{fig:2}
\end{figure}\\
Полная деформация образца в момент времени $t'$ определяется суммой упругой дефомации и дефомации ползучести согласно
соотношению.
\begin{align}
    \varepsilon(t') = \frac{\sigma}{E} + G(\sigma, t_{0}, t')
\end{align}
Последнее слагаемое называется функцией ползучести, для каждого материала может быть своей, для некоторых материалов зависит от гистерезиса.
При линейной деформации деформация ползучести может быть найдена как произведение двух функций : одна из них зависит только от времени,
вторая - только от напряжений.
\begin{align}
    \varepsilon(t') = \frac{\sigma}{E} + G(t_{0}, t')\sigma
\end{align}
У нестареющих материалов функция удлинения от времени включает функцию разностного типа.
\begin{align}
    \varepsilon(t) = \frac{\sigma}{E} + G(t - t_{0})\sigma
\end{align}
Для получения зависимостей между переменными напряжениями и деформациями важное значение имеет принцип суперпозиции Больцмана.
Согласно ему суммарная деформация ползучести при переменном напряжении может быть найдена как сумма деформаций ползучести, вызванных
соответствующими приращениями напряжений. Если нагрузка является ступенчатой, то можно записать
\begin{align}
    \varepsilon(t_{4}) = \frac{\sigma(t_{4})}{E} + G(t_{4} - t_{0})\Delta\sigma_{1} + ... + G(t_{4} - t_{3})\Delta\sigma_{4}
\end{align}
Если напряжение протекает по непрерывной кривой, то формула превращается в интегральное уравнение, вообще говоря, наследственного типа.
\begin{align}
    \varepsilon(t) = \frac{\sigma}{E} + \int_{0}^{t}{G(t - \tau)d\sigma(\tau)}
\end{align}
Определение деформаций по этой формуле учитывает всю историю изменения напряжений. Для напряжений в простейшем случае функцию ползучести можно представить 
в виде.
\begin{align}
    G(t - \tau) = A - A \exp(-\alpha (t - \tau))
\end{align}
Где A - вещественное число.\\
Если в некоторый момент времени образец начать разгружать, то накопленная за долгое время деформация ползучести начнет уменьшаться. Этот
процесс называется релаксацией деформации. Релаксационные свойства при линейной ползучести описываются выражением.
\begin{align}
    \sigma(t) = \varepsilon(t)E - \int_{0}^{t}{R(t - \tau)\varepsilon(\tau)d\tau}
\end{align}
Здесь вводится функция $R(t - \tau)$, которая называется ядром релаксации.
\section{Обработка данных}
\begin{enumerate}
    \item Данные материала, к которому прикладывается напряжение.
    \begin{table}[htbp]
        \centering
        \begin{tabular}{|c|c|c|}
            \hline
            Толщина, мм  & Ширина, мм & Высота, мм\\
            \hline
            3.00 & 15.00 & 80.00\\
            \hline
        \end{tabular}
        \label{tab:1}
    \end{table}
    \item Снимаем зависимость длины материала от времени, во время которого происходит нагрузка.
    \begin{table}[htbp]
        \centering
        \begin{tabular}{|c|c|c|c|c|c|}
            \hline
        Длина, мкм  & Время, с & Длина, мкм & Время, с & Длина, мкм & Время, с\\
            \hline
            \multicolumn{2}{|c|}{По 5 с} & \multicolumn{2}{|c|}{По 10 - 20 с} & \multicolumn{2}{|c|}{По 30 с}\\
            \hline
            48 & 5 & 127 & 10 & 185 & 0\\
            65 & 10 & 132 & 20 & 191 & 30\\
            75 & 15 & 137 & 30 & 195 & 60\\
            83 & 20 & 140 & 40 & 199 & 90\\
            90 & 25 & 144 & 50 & 203 & 120\\
            96 & 30 & 148 & 60 & 205 & 150\\
            103 & 35 & 155 & 80 & 209 & 180\\
            107 & 40 & 159 & 120 & 213 & 210\\
            110 & 45 & 165 & 140 & 215 & 240\\
            114 & 50 & 169 & 160 & 218 & 270\\
            118 & 55 & 173 & 180 & 222 & 300\\
            122 & 60 & 175 & 200 & 224 & 330\\
            \hline
        \end{tabular}
    \end{table}
    \item График относительного удлинения от времени.
\end{enumerate}
\section{Вывод}
\end{document}