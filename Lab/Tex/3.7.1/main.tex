\documentclass[a4paper, 12pt]{article}
\usepackage[a4paper, top = 1.5cm, bottom = 1.5cm, left = 1cm, right = 1cm]{geometry}
\usepackage[english, russian]{babel}
\usepackage{graphicx}
\usepackage{subcaption}
\usepackage{mathtools}
\usepackage{amsfonts}
\usepackage{float}
\usepackage{wrapfig}
\title{Лабораторная работа "Изучение скин-эффекта в полом цилиндре"}
\author{Шевцов Кирил Б03-402}
\begin{document}
\maketitle
\section{Скин-эффект, толщина скин слоя}
Рассмотрим квазистационарное поле внутри проводящей среды в простейшем плоском случае.
Пусть вектор $\vec{E}$ направлен всюду вдоль оси $y$ и зависит только от координаты $x$, т. е. ${E_x} = {E_z} \equiv 0$, $E_y=E_y(x,t)$.
Положим малое изменение электрического поля, то есть $D'_{t} = 0$. Тогда уравнение Максвелла перепишется в виде.
\begin{equation*}
    \nabla \times \vec{H} = \sigma \vec{E} = rot\vec{H}
\end{equation*}
Что равносильно уравнению, после взятия ротора обеих частей.
\begin{equation}
    \nabla^2\vec{H}=\sigma\mu\mu_0 \frac{\partial \vec{H}}{\partial t}
\end{equation}
Называемым волновым уравнением. Аналогично для вектора электрического поля
\begin{equation}
    \frac{\partial^2 E_y}{\partial x^2} = \sigma\mu\mu_0\frac{\partial E_y}{\partial t}
\end{equation}
% Если $E_y(0,t)=E_0 e^{i\omega t}$ то решением (\ref{eq:diffusion_chastni}) будет функция вида
\begin{equation}
    E_y(x,t)=E_0 e^{-x/\delta} e^{i(\omega t - x/\delta)}
\end{equation}
где величина, обозначенная как $\delta$
\begin{equation}
    \delta = \sqrt{\frac{2}{\omega\sigma\mu\mu_0}}
\end{equation}
Называется толщиной скин-слоя. Электрическое поле способно проникать на некоторую толщину в проводник, а затем быстро затухать
по экспоненциальному закону.
Из соображении симметрии и непрерывности соответствующих компонет векторов $\vec{E}$ и $\vec{H}$ можем сказать что
\begin{equation}
    H_z = H(r)e^{i\omega t} \text{, } E_\varphi = E(r)e^{i\omega t}
\end{equation}
Где функции $H(r)$ и $E(r)$ непрерывны. Внутри цилиндра токов нет, следовательно $H(r)=H_1=\text{const}$ внутри цилиндра.
По теореме об электромагнитной индукции
\begin{equation*}
    E(r) = -\frac{1}{2}\mu_0 r \cdot i \omega H_1
\end{equation*}
откуда мы получаем граничное условие
\begin{equation}
    E_1=E(a)= -\frac{1}{2}\mu_0 a \cdot i \omega H_1
\end{equation}
В приближении $h \ll a$ можем пренебречь кривизной стенки и смоделировать 
его бесконечной полосой. Тогда, надо решить уравнение с граничными условиями. Решая уравнение получим связь полей $H_1$ 
(поле внутри цилиндра которое мы будем измерять) и $H_0$, которое колебается с частотой $\omega$
\begin{equation}
    H_1 = \frac{H_0}{\ch(\alpha h) + \frac{1}{2} \alpha a \sh(\alpha h)} 
    \quad
    \alpha = \sqrt{i\omega \sigma \mu_0} = \frac{\sqrt{2}}{\delta}e^{i\pi/4}
\end{equation}
Из этой формулы получим сколько по фазе отстает поле $H_1$ от $H_0$. При $\delta \ll h$
(на высоких частотах)
\begin{equation}
    \psi \approx \frac{\pi}{4} + \frac{h}{\delta} = 
    \frac{\pi}{4} + h \sqrt{\frac{\omega \sigma \mu_0}{2}}
\end{equation}
При $\delta \gg h$ (низкие частоты)
\begin{equation}
    \tg \psi \approx \frac{ah}{\delta^2} = \pi a h \sigma \mu \mu_0 \nu
\end{equation}
\section{Лабораторная установка}
\begin{wrapfigure}{l}{0.4\textwidth}
    \begin{center}
        \includegraphics[width=0.4\textwidth]{equip.png}
    \end{center}
    \label{fig:1}
\end{wrapfigure}
Переменное магнитное поле создается соленоидом 1, на который подается переменный ток со звукового генератора ЗГ. Внутри соленоида расположен медный экран 2. Магнитное поле внутри цилиндра измеряется катушкой 3. Напряжение на катушке пропорциональна производной $\dot{B_1}(t)$
\begin{equation}
    U(t) \propto \dot{B_1}(t) = -i\omega H_1 e^{i\omega t}
\end{equation}
Поле внутри цилиндра пропорциональна току через соленоид
\begin{equation}
    H_0(t) \propto I(t)
\end{equation}
Отсюда несложно увидеть, что
\begin{equation}
    \frac{\left|H_{1}\right|}{\left|H_{0}\right|} = c \cdot \frac{U}{\nu I} = \xi_0 \xi
\end{equation}
где константу  $\xi_0$ можно определить из условия  $\frac{\left|H_{1}\right|}{\left|H_{0}\right|} \rightarrow 1$ при
$\nu \rightarrow 0$.\\
При измерениях разности фаз нужно учесть, что измеренная на осциллографе разность фаз $\varphi$ будет на $\frac{\pi}{2}$ больше реальной $\psi$:
\[\varphi = \psi + \frac{\pi}{2}\]
\section{Выполнение работы}
Параметры установки: $2a = 45$ мм - длина стенок, $h=1.5$ мм - толщина стенок. Проводимость материала $\sigma = 5\cdot 10^7$ См/м. Частота, при которой
глубина проникновения равна толщине стенок цилиндра $\nu_h = 2251,6$ Гц.
При низких частотах толщина скин-слоя превосходит толщину образца $\delta \gg h$. В таком случае
\begin{equation}
    \left(\frac{|H_1|}{|H_0|}\right)^2 = (\xi_0\xi)^2 \approx \frac{1}{1+\left(\frac{ah}{\delta^2}\right)^2} = \frac{1}{1 + \left(\pi ah\nu\mu_0\sigma\right)^2}
\end{equation}
Тогда:
\begin{equation}
    \frac{1}{\xi^2}=\xi_0^2B^2\nu^2 + \xi_0^2 \text{, где } B=\pi a h \sigma \mu_0
\end{equation}
\begin{figure}[H]
    \centering
    \small
    \includegraphics[width=0.7\textwidth]{sigma1.png}
    \newpage
\end{figure}
Из графика значения: $\xi_0^2B^2 = 1,927\cdot 10^{-7}, \xi_{0}^{2} = 0,005$ тогда:
\[\sigma = 4.6 \cdot 10^7 \ \frac{\text{См}}{\text{м}}  \]
Полученное значение - типичное значение проводимости меди при комнатной температуре.
\section{Измерение проводимости через разность фаз при низких частотах}
Измерим сдвиг фаз $\psi$ при малых частотах.
\begin{table}[htbp]
    \centering
    \label{tab:hfr}
        \begin{tabular}{|c|c|c|c|c|c|}
        \hline
        $\nu$, Гц & $\varphi$, рад & $U$, В & $I$, мА & $x_0$, дел & $x$, дел \\
        \hline
        112.5500 & 0.8976 & 0.5232 & 400.2000 & 3.5 & 1.0 \\
        135.0600 & 0.8378 & 0.5710 & 387.5400 & 3.0 & 0.8 \\
        157.5700 & 0.7250 & 0.6070 & 376.6800 & 2.6 & 0.6 \\
        180.0800 & 0.5712 & 0.6330 & 367.5600 & 2.2 & 0.4 \\
        202.5900 & 0.5498 & 0.6536 & 359.9000 & 2.0 & 0.35 \\
        225.1000 & 0.5236 & 0.6690 & 353.5000 & 1.8 & 0.3 \\
        353.7000 & 0.5236 & 0.7060 & 330.5500 & 1.2 & 0.2 \\
        482.3000 & 0.3927 & 0.7080 & 317.5000 & 0.8 & 0.1 \\
        610.9000 & 0.3696 & 0.7010 & 307.1500 & 3.4 & 0.4 \\
        739.5000 & 0.2244 & 0.6860 & 297.4500 & 2.8 & 0.2 \\
        868.1000 & 0.0714 & 0.6680 & 287.8200 & 4.4 & 0.1 \\
        \hline
        \end{tabular}
\end{table}
Построим график $\tg{\psi} (\nu)$ малых значений частоты. По формуле ниже
\begin{equation}
    \tg \psi = \frac{ahw \sigma \mu_0}{2} = \pi ah\mu_0\sigma \nu \ \ (\mu = 1)
\end{equation}
Коэффициент наклона прямой: \[\pi ah \mu_0\sigma = k = 9,3 \cdot 10^{-3} \ \text{с}\]
\[\sigma = \frac{k}{\pi ah \mu_0} = 3,37 \cdot 10^7 \ \frac{\text{См}}{\text{м}}\]
\begin{figure}[H]
    \centering
    \includegraphics[width=0.7\textwidth]{sigma2.png}
    \newpage
\end{figure}
Полученное значение близко к реальному значению проводимости меди при комнатоной температуре.
\section{Измерение проводимости через разность фаз в высокочастотном диапазоне}
\begin{equation}
    \psi - \pi/4 = k\cdot \sqrt{\nu}; \ k = h\sqrt{\pi\mu_0\sigma}
\end{equation}
Получено значение $k = 0.0184 \pm 0.0014$, отсюда получаем значение проводимости:
\begin{equation}
    \sigma = 3.520 \cdot 10^7 \ \frac{\text{См}}{\text{м}}
\end{equation}
График $\psi - \pi/4 = f(\sqrt{\nu})$
\begin{figure}[htbp]
    \centering
    \includegraphics[width=0.7\textwidth]{sigma3.png}
\end{figure}\\
Измерим зависимость индуктичности катушки от частоты с помощью $RCL$-метра:
\begin{table}[htbp]
    \centering
    \begin{tabular}{|c|c|c|c|c|c|c|c|c|}
        \hline
        $\nu$, кГц & 1 & 1,5 & 2 & 2,5 & 3 & 4 & 15 & 20\\
        $L$, мГн & 13,79 & 2,9 & 2,92 & 2,90& 2,89& 2,89 & 3,21& 3,71\\
        \hline
    \end{tabular}
\end{table}
График $L(\nu)$:
\begin{figure}[H]
    \centering
    \includegraphics[width = 0.7\textwidth]{L_nu.png}
\end{figure}
\section{Вывод}
В данной лабораторной работе измерена удельная проводимость меди 3-мя различными способами с помощью явления скин-эффекта.
\begin{table}[htbp]
    \begin{center}
        \begin{tabular}{|l|c|c|}
            \hline
            Метод измерения & $\sigma, 10^{7} \ \frac{\text{См}}{\text{м}}$ & $\varepsilon_{\sigma}$, $\%$\\
            \hline
            Через токи и напряжения & 4,6 & 8\\
            Разности фаз (низкие частоты) & 3,37 & 32,6\\
            Разности фаз (высокие частоты) & 3,52 & 29,6\\
            \hline
        \end{tabular}
    \end{center}
\end{table}
\end{document}