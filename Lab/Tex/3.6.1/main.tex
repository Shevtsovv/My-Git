\documentclass[a4paper, 12pt]{article}
\usepackage[a4paper, top = 1.5cm, bottom = 1.5cm, left = 1cm, right = 1cm]{geometry}
\usepackage[english, russian]{babel}
\usepackage{graphicx}
\usepackage{subcaption}
\usepackage{mathtools}
\usepackage{amsfonts}
\usepackage{float}
\title{Лабораторная работа № 3.6.1 "Спектральный анализ электрических сигналов"}
\author{Кирилл Шевцов Б03-402}
\begin{document}
\maketitle
Задача об изучении спектров сигналов сводится к поиску отклика системы $g(t)$ на внешнее воздействие $f(t)$.
\begin{enumerate}
    \item \textbf{Исследование спектра периодических последовательностей импульсов.}\\
    Пусть на вход последовательной RLC-цепочки подается периодическая последовательностей импульсов, с длительностью $\tau$
    и периодом $T$. Изобразим спектры этого сигнала, изменяя длительность сигнала в диапазоне $50 \mu s - 200 \mu s$.
    \begin{figure}[htbp]
        \centering
        \begin{subfigure}[b]{0.4\textwidth}
            \centering
            \includegraphics[width=\textwidth]{1a.png}
            \caption{50 мкс}
            \label{fig:sub1}
        \end{subfigure}
        \begin{subfigure}[b]{0.4\textwidth}
            \centering
            \includegraphics[width=\textwidth]{1b.png}
            \caption{100 мкс}
            \label{fig:sub2}
        \end{subfigure}
        \begin{subfigure}[b]{0.4\textwidth}
            \centering
            \includegraphics[width=\textwidth]{1c.png}
            \caption{150 мкс}
        \end{subfigure}
        \begin{subfigure}[b]{0.4\textwidth}
            \centering
            \includegraphics[width=\textwidth]{1d.png}
            \caption{200 мкс}
        \end{subfigure}
        \caption{Спектры при переменном $\tau$}
        \label{fig:main}
    \end{figure}\\
    \textbf{Вывод} Периодическая последовательность импульсов имеет спектр состоящий из набора полос на частотах $k\Omega$, а ширина этого
    спектра меняется при изменении длительности сигнала - уменьшается при увеличении времени и уменьшается в противном случае таким образом,
    что произведение $\tau \Omega \sim 2\pi$
    \item \textbf{Исследование периодической последовательности цугов гармонических колебаний}
    Цуг - это "обрывок" синусоиды / косинусоиды. Получим спектр этого сигнала, изменяя частоту повторения импульсов.
    \begin{figure}[htbp]
        \centering
        \begin{subfigure}[b]{0.4\textwidth}
            \centering
            \includegraphics[width=\textwidth]{zug1.png}
            \caption{2 кГц}
            \label{fig:zug1}
        \end{subfigure}
        \begin{subfigure}[b]{0.4\textwidth}
            \centering
            \includegraphics[width=\textwidth]{zug2.png}
            \caption{4 кГц}
            \label{fig:zug2}
        \end{subfigure}
        \begin{subfigure}[b]{0.4\textwidth}
            \centering
            \includegraphics[width=\textwidth]{zug3.png}
            \caption{6 кГц}
            \label{fig:zug3}
        \end{subfigure}
        \begin{subfigure}[b]{0.4\textwidth}
            \centering
            \includegraphics[width=\textwidth]{zug4.png}
            \caption{8 кГц}
            \label{fig:zug4}
        \end{subfigure}
        \caption{Спектры при переменном $\tau$}
        \label{fig:zug_all}
    \end{figure}\\
    На фотографиях спектра видно, что ширина цуга неизменна и равна $\nu_{0} = 1/\tau = 20\ \text{кГц}$, частота самого большого пика соответсвует
    частоте несущей.
    \item \textbf{Исследование спектров гармонических сигналов, модулированных по амплитуде}\\
    Говорят, что сигнал модулирован по амплитуде, если
    \begin{equation}
        f(t) = a_{0}\left[1 + m\cos(\Omega t)\right]\sin(\omega t) = a(t)\sin(\omega t)
    \end{equation}
    Для амплитудно-модулированного сигнала делают $\Omega \ll \omega$ (так как интересно посмотреть на амплитуды 
    близ лежащих частот), $m \ll 1$. Получим спектр амплитудно модулированного сигнала.
    \begin{align*}
        f(t) = a_{0}\sin(\omega t) + a_{0}m\cos(\Omega t)\sin(\omega t) =
        a_{0}\sin(\omega t) + \frac{a_{0}m}{2}\sin((\omega - \Omega) t) + \frac{a_{0}m}{2}\sin((\omega + \Omega) t)
    \end{align*}
    Аналогичные рассуждения, если модулированный по амплитуде сигнал содержит $\cos(\omega t)$.
    Сигнал, модулированный по амплитуде, очень сильно осциллирует поэтому его детектируют, усредняя квадрат сигнала, а затем проводят фильтрацию.
    \begin{align*}
        f^{2}(t) = a^{2}_{0}\left[1 + m\cos(\Omega t)\right]^{2}\sin^{2}(\omega t) = \frac{a^{2}(t)}{2}\left[1 + 2m\cos(\Omega t)\right] + \Sigma
    \end{align*}
    Изобразим ниже модулированные по амплитуде сигналы при разных значених глубины модуляции.
    При изменении параметра глубины модуляции отношение амплитуд боковых и центральных частот меняется
    по какому-то закону.
    \begin{figure}[htbp]
        \centering
        \begin{subfigure}[b]{0.3\textwidth}
            \centering
            \includegraphics[width=\textwidth]{am.png}
            \caption{m = 0.05}
            \label{fig:m1}
        \end{subfigure}
        \begin{subfigure}[b]{0.3\textwidth}
            \centering
            \includegraphics[width=\textwidth]{20m.png}
            \caption{m = 0.1}
            \label{fig:m2}
        \end{subfigure}
        \begin{subfigure}[b]{0.3\textwidth}
            \centering
            \includegraphics[width=\textwidth]{50m.png}
            \caption{m = 0.25}
            \label{fig:m3}
        \end{subfigure}
        \caption{АМ-сигналы при разной глубине модуляции}
    \end{figure}\\
    Измерим отношение боковой и центральной амплитуды спектра, построим график $a_{1}/a_{2}(m)$
    \begin{table}[htbp]
        \centering
        \begin{tabular}{|c|c|c|c|c|c|c|c|c|c|c|}
            \hline
            Глубина модуляции $m$ & 0.05 & 0.1 & 0.15 & 0.2 & 0.25 & 0.3 & 0.35 & 0.4 & 0.45 & 0.5\\
            $a_{1}/a_{2}(m)$ & 20.12 & 10.49 & 7.13 & 5.31 & 4.18 & 3.52 & 2.95 & 2.61 & 2.30 & 2.075\\
            \hline
        \end{tabular}
        \caption{Отношение боковой и центральной амплитуд при разных $m$}
    \end{table}
    Глубина модуляции выбрана $m = 0.5$, несущая частота равна $\nu = 50$ кГц.
    \begin{figure}[htbp]
        \centering
        \includegraphics[width=0.7\linewidth]{am_approx.png}
        \caption{Зависимость отношения амплитуд от глубины модуляции}
        \label{fig:a1a2}
    \end{figure}\\
    \textbf{Вывод} Амплитудно модулированный сигнал имеет в спектре три полагающие частоты: несущую и несущую $\pm$ модулируемую.
    Детектирование АМ-сигнала происходит по усреднению квадрата этого сигнала, а фильтр помогает извлечь информацию, которую переносил этот сигнал.
    Отношение боковой и центральной амплитуды спектра (именно в таком порядке) равно величине глубины модуляции.
    \item \textbf{RC-фильтр}\\
    Фильтры используются, как было сказано ранее, для получения сигнала на основных частотах, и для подавления так называемых
    нежелательных частот. Для фильтрации используют линейные преобразователи, с помощью которых можно получить функцию отклика системы на
    какое-либо воздействие. Таковыми, для примера, могут являться: RC-фильтр, и RL-фильтр.\\
    Рассмотрим RC-фильтр. Получим выражение функции отклика системы $\lambda(\Omega)$ для входного сигнала $f(t)$ для напряжения на конденсаторе.
    \begin{align*}
        \frac{q(t)}{C} + JR = \varepsilon_{0}\cos(\Omega t) \rightarrow U_{c} + RC\frac{dU_{c}}{dt} = \varepsilon_{0}\cos(\Omega t)\\
        \tilde{U_{c}} + RC\frac{d\tilde{U_{c}}}{dt} = \varepsilon_{0} \exp(j\Omega t)\quad (\tilde{U_{c}} = \tilde{U_{0}}e^{j\Omega t})\quad \tilde{U_{0}} = \frac{\varepsilon_{0}}{1 + RCj\Omega}\\
    \end{align*}
    Откуда получим выражение для выходного сигнала
    \begin{align*}
        \tilde{U_{c}} = \frac{1}{1 + j\Omega RC}\varepsilon_{0}\exp(j\Omega t) = \lambda(\Omega)g(t)
    \end{align*}
    Отсюда видно, что RC-фильтр является фильтром низких частот, поскольку большие частоты не дают обнаружить $g(t)$. В технике это называется Low-Pass Filter.\\
    \textbf{Замечание} Можно показать, что RL-цепочка - это фильтр высоких частот.
    Подсоединим RC-фильтр к второму каналу осциллографа, исследуем сигнал исходный и тот, который получается фильтрованием для разный частот.
    Длительность импульса равна $\tau = 12.5 \mu s $, частоты $\nu = 80, 330$ кГц.
    % \begin{figure}[H]
    %     \centering
    %     \includegraphics[width=0.5\linewidth]{rc.png}
    %     \caption{RC-фильтр}
    %     \label{fig:rc}
    % \end{figure}
    \begin{figure}[htbp]
        \centering
        \begin{subfigure}[b]{0.6\textwidth}
            \centering
            \includegraphics[width=\textwidth]{rc1.png}
            \caption{}
            \label{fig:rc1}
        \end{subfigure}
        \begin{subfigure}[b]{0.6\textwidth}
            \centering
            \includegraphics[width=\textwidth]{rc2.png}
            \caption{}
            \label{fig:rc2}
        \end{subfigure}
    \end{figure}\\
    \textbf{Вывод} RC-фильтр убирает высокие частоты, делая срез, где наблюдается снижение красного спектра.
    В первом случае фильтр делает срез частоты примерно на 8,6 МГц, на втором графике - примерно на 7 МГц - линии спектра на частотах
    выше не детектируются сигналом, они лишние. RC- фильтр - действительно фильтр низких частот.
\end{enumerate}
% \textbf{Определение параметров RC-фильтр}
% \begin{enumerate}
%     % \item \textbf{/home/amisto/Рабочий стол/My-Factory/am_approx.pngЧастота среза $f_c$} — определяется по уровню $-3$ дБ от полки низких частот на канале B.
%     % Если максимум на низких частотах $A_0$, то находим частоту, где амплитуда 
%     % \[
%     % A(f_c) \approx 0.707 A_0
%     % \]
%     % \item \textbf{Коэффициент передачи на НЧ} — отношение амплитуд Канал B / Канал A на низких частотах:
%     % \[
%     % K_{\text{НЧ}} = \frac{A_{\text{вых}}}{A_{\text{вх}}} \quad \text{при} \quad f \ll f_c
%     % \]
%     % \item \textbf{Подавление на высоких частотах} — видно, что после $f_c$ спад $\sim 20$ дБ/дек (характерно для RC-фильтра 1-го порядка):
%     % \[
%     % \left| H(f) \right| = \frac{1}{\sqrt{1 + \left( \frac{f}{f_c} \right)^2 }} \quad \text{при} \quad f > f_c
%     % \]
% \end{enumerate}
\textbf{Вывод}\\
В работе были изучены спектры электрических сигналов, таких как периодическая последовательность импульсов, цуги и спектр
амплитудно - модулируемого сигнала.
\end{document}