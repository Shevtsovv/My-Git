\documentclass[a4paper, 12pt]{article}
\usepackage[a4paper, top = 1.5cm, bottom = 1.5cm, left = 1cm, right = 1cm]{geometry}
\usepackage[english, russian]{babel}
\usepackage{graphicx}
\usepackage{subcaption}
\usepackage{mathtools}
\usepackage{amsfonts}
\usepackage{float}
\usepackage{multirow}
\title{Лабораторная работа № 3.6.1 "Петля гистерезиса - динамический метод"}
\author{Кирилл Шевцов Б03-402}
\begin{document}
\maketitle
\section{Петли гистерезиса}
\begin{enumerate}
    \item Параметры установки: $R_0 = 0,22$ Ом, $R_u = 20$ кОм, $C_u = 20$ мкФ.
    \item Характеристики образцов.
    \begin{table}[H]
    \centering
        \begin{tabular}{|c|c|c|c|}
        \hline
        Параметр & Пермаллой & Кремнистое железо & Феррит\\
        $N_{0}$, витков & 15 & 20 & 45\\
        $N_{i}$, витков & 300 & 200 & 400\\
        $S$, см$^2$ & 0,66 & 2,0 & 3,0\\
        $2\pi R$, см & 14,1 & 11,0 & 25,0\\
        \hline
        \end{tabular}
        \label{tab:1}
    \end{table}
    \item Калибровка осциллографа, значение тока для получения петли гистерезиса,
    картины петель гистерезиса для трех образцов.
    \begin{table}[H]
    \centering
        \begin{tabular}{|c|c|c|c|}
            \hline
            Параметр & Пермаллой & Кремнистое железо & Феррит\\
            $2x$, дел & $7,8 \pm 0,1$ & $5,6 \pm 0,1$ & $6,0 \pm 0,1$\\
            $2y$, дел & $3,8 \pm 0,1$ & $7,0\pm 0,1$ & $6,0 \pm 0,1$\\
            $I_{\text{эф}}$, мА & $257,8 \pm 0,1$ & $1004,3 \pm 0,1$ & $549,9 \pm 0,1$\\
            $K_X$, В/дел & 0,02 & 1,0 & 0,05\\
            $K_Y$, В/дел & 0,05 & 0,02 & 0,02\\
            \hline
            Петля гистерезиса &
            \raisebox{-\totalheight}{\includegraphics[width=0.2\linewidth]{hyst1.jpeg}} &
            \raisebox{-\totalheight}{\includegraphics[width=0.2\linewidth]{hyst.png}} &
            \raisebox{-\totalheight}{\includegraphics[width=0.25\linewidth]{hyst2.jpg}} \\
            \hline
        \end{tabular}
        \label{tab:2}
    \end{table}
    \section{Калибровка осциллографа, расчет постоянной $\tau$, полей $B_{s}$ и $H_{c}$}
    \item Отключаем намагничивающую обмотку от цепи, подсоединяем оба провода к одной из ее клемм.
    Чувствительность экрана $W = 2\sqrt{2}R_{0}I_{\text{эфф}}/(2x)$
    \begin{table}[H]
        \centering
            \begin{tabular}{|c|c|c|c|}
            \hline
            Параметр & Пермаллой & Кремнистое железо & Феррит \\
            $I_{\text{эфф}}$, А & $0,248 \pm 0,001$ & $1,043 \pm 0,001$ & $0,525 \pm 0,001$ \\
            $2x$, дел & $6,0 \pm 0,1$ & $7,0 \pm 0,1$ & $10,0 \pm 0,1$ \\
            $W$, В/дел & $0,025 \pm 0,001$ & $0,092 \pm 0,001$ & $0,032 \pm 0,001$ \\
            $K_X$, В/дел & 0,02 & 1,00 & 0,05 \\
            \hline
            \end{tabular}
            \label{tab:3}
    \end{table}
    \item Расчитаем постоянную времени $\tau$, $U_{\text{вх}}$ с частотой $\nu = \omega/2\pi = 50$ Гц.\\
    Входное и выходное напряжение.
    \begin{align}
        U_{\text{вх}} = 2y\cdot K_{Y} \quad U_{\text{вых}} = 2x\cdot K_{X}
    \end{align}
    где $K_{X}$, $K_{Y}$ - чувствительности каналов $X$ и $Y$ осциллографа. Постоянная времени $\tau = RC = (U_{\text{вх}})/(\omega\cdot U_{\text{вых}})$
    \begin{table}[H]
        \centering
            \begin{tabular}{|c|c|}
            \hline
            Параметр & Значение\\
            $2y$, дел & $8,0 \pm 0,1$\\
            $2x$, дел & $6,2 \pm 0,1$\\
            $K_Y$, В/дел & 2\\
            $K_X$, В/дел & 0,02\\
            $U_{\text{вх}}$, В & $16,0 \pm 0,1$\\
            $U_{\text{вых}}$, В & $0,124 \pm 0,001$\\
            $\tau$, с & $0,41 \pm 0,01$\\
            \hline
            \end{tabular}
            \label{tab:4}
    \end{table}
    \item Полученные данные. Коэрцетивная сила, магниное поле образцов можно вычислить согласно соотношениям
    \begin{align}
        H_{c} = \frac{IN_{0}}{2\pi R} = \frac{N_{0}}{2\pi R}\frac{K_{X}}{R_{0}} \quad B_s = \frac{R_{u}C_{u}U_{\text{вых}}}{SN_{u}} = \frac{R_{u}C_{u}}{SN_{u}}2y\cdot K_{Y}
    \end{align}
    \begin{table}[H]
        \centering
        \begin{tabular}{|c|c|c|}
            \hline
            Материал & Параметр & Эксперимент \\
            \hline
            \multirow{2}{*}{Пермаллой}
            & $H_c$, А/м & $9,67 \pm 1,00$ \\
            & $B_s$, Тл & $1,01 \pm 0,01$  \\
            \hline
            \multirow{2}{*}{Кремнистое железо}
            & $H_c$, А/м & $826,4 \pm 1,0$ \\
            & $B_s$, Тл & $0,20 \pm 0,01$ \\
            \hline
            \multirow{2}{*}{Феррит}
            & $H_c$, А/м & $16,3 \pm 1,0$ \\
            & $B_s$, Тл & $0,067 \pm 0,001$\\
            \hline
        \end{tabular}
    \end{table}
    Видно, что пермаллой и феррит - это ферромагнетики, которые имеют небольшую коэрцетивную силу, то есть их размагнитит небольшая
    напряженность магнитного поля $H_{c}$. Кремнистое железо имеет огромную коэрцетивную силу - именно по этому его называют "жестким"
    ферромагнетиком.
\end{enumerate}
\section{Вывод}
В ходе лабораторной работы были экспериментально исследованы магнитные свойства трёх
ферромагнитных материалов. Получены петли гистерезиса ферромагнитных материалов с помощью осциллографа.
\end{document}