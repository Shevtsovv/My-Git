\documentclass[a4paper, 12pt]{article}
\usepackage[a4paper, top = 1.5cm, bottom = 1.5cm, left = 1cm, right = 1cm]{geometry}
\usepackage[english, russian]{babel}
\usepackage{graphicx}
\usepackage{subcaption}
\usepackage{mathtools}
\usepackage{amsfonts}
\usepackage{float}
\usepackage{wrapfig}
\usepackage{multirow}
\title{Лабораторная работа 3.2.4-5 "Свободные и вынужденные колебания"}
\author{Шевцов Кирилл Б03-402}
\begin{document}
\maketitle
\section{Свободные колебания}
\begin{wrapfigure}{r}{0.4\textwidth}
    \includegraphics[width=\linewidth]{oscilla.png}
    \label{fig:oscilla}
\end{wrapfigure}
Рассмотрим колебательный контур, состоящий из конденстатора и катушки индуктивностью $L$. Согласно второму правилу Кирхгофа.
Уравнение вида
\begin{align}
     L\frac{dJ}{dt} + \frac{q}{C} = 0 \Rightarrow L\frac{d^{2}q}{dt^{2}} + \frac{q}{C} = 0
\end{align}
Описывает свободные гармонические колебания в LC-контуре. Решение этого уравнения
\begin{align}
     q(t) = A\cos(\omega t) + B\sin(\omega t)
\end{align}
Где константы $A, B$ определяются начальными условиями: $q(0) = q_{0},\ q'(0) = J_{0}$.
Теперь рассмотрим электрический контур, состоящий из последовательно соединённых конденстора $C$, катушки индуктивности $L$ и резистора $R$. Обозначим разность потенциалов на
конденсаторе $U_C$, а ток, текущий в контуре, через $I$. Второе правило Кирхгофа:
\begin{equation}
     L \dfrac{d^2q}{dt^2}+R\dfrac{dq}{dt}+\dfrac{q}{C}=0.
\end{equation}
Вводя обозначения $\gamma = \dfrac{R}{2L}$, $\omega_0^2=\dfrac{1}{LC}$, получим уравнение
\begin{equation}
     \ddot{I}+2\gamma\dot{I}+\omega_0^2I=0.
\end{equation}
Решения этого уравнения полезно исследовать.
\section{Затухающие колебания}
Рассмотрим уравнение затухающих колебаний. Его характеристическое уравнение
\begin{align}
     \lambda^{2} + 2\gamma \lambda + \omega_{0} = 0 \Rightarrow \sqrt{D} = 2\sqrt{\gamma^{2} - \omega^{2}_{0}}
\end{align}
В случае, когда $\gamma < \omega_0$, имеем $\kappa = i\omega$, где $\omega = \sqrt{\omega_0^2 - \gamma^2}$ - частоты свободных (собственных) колебаний. Тогда ток ($\lambda = \gamma \pm \omega i$)
 \begin{equation}
     J(t) = J_{0}\exp(-\gamma t)\left[A\sin(\omega t) + B\cos(\omega t)\right]
 \end{equation}
затухает и имеет колебательный характер. Величина $\gamma$ определяет характеристическое время затухания колебаний: $\gamma = \dfrac{1}{\tau}$, где $\tau$ - время затухание амплитуды в $e$ раз.
\section{Апериодические колебания}
В случае $\gamma > \omega_0$ режим колебаний называется апериодическим. Решение дифференциального уравнения:
\begin{align}
     q(t) = A\exp^{\lambda_{1}t} + B\exp^{\lambda_{2}t}
\end{align}
Процесс в этом случае не является колебательным. Режим, соответствующий $\gamma = \omega_0$, называются критическим. В этом случае предельный переход $\omega \rightarrow 0$ в $(5)$ даст
\begin{equation}
     R_{\text{кр}}= 2 \sqrt{\dfrac{L}{C}}
\end{equation}
называется критическим сопротивлением контура.\\
Добротностью контура по определению называют
\begin{align}
     Q = \frac{\pi}{\Theta}
\end{align}
где $\Theta$ - логарифмический декремент затухания. В таком случае, если $\gamma \ll \omega_{0}$, то несколько преобразований
\begin{align}
     \frac{dW}{dt} = \frac{\left|\delta W\right|}{T} = 2\gamma W \Rightarrow \frac{W}{\left|\delta W\right|} = \frac{1}{2\gamma T} = \frac{1}{2\Theta} = \frac{Q}{2\pi} \Rightarrow Q = 2\pi\frac{W}{\left|\delta W\right|}
\end{align}
дают энергетический смысл добротности - отношение запасенной энергии к убыли энергии за период.
\section{Вынужденные колебания}
При наличии источника, который настроен на подачу сигнала с частотой $\Omega$, уравнение колебаний запишется в виде
\begin{align}
     L \dfrac{d^2q}{dt^2}+R\dfrac{dq}{dt}+\dfrac{q}{C} = U_{0}\sin(\Omega t).
\end{align}
Его решение получается методом комплесных амплитуд. Для этого пишется напряжение на конденсаторе в виде $U_{C}$. Пусть $U_{C} = U_{0}\exp(i\Omega t + \varphi),\ U(t) = U_{0}\exp(i\Omega t)$ а затем решаются аналогично предыдущим
пунктам.
\section{Выполнение работы}
\subsection{Измерение периодов свободных колебаний}
Измерим индуктивность $L$ и сопротивление катушки $R_L$ в зависимости от частоты. Получим $L = 200 \pm 0,2$ мГн.
\begin{table}[htbp]
     \centering
     \begin{tabular}{|c|c|c|}
          \hline
          $\nu$, Гц & $L$, мГн & $R_L$, Ом \\
          \hline
          50 & 200,4 & 11,1\\
          1000 & 200,1 & 18,8\\
          5000 & 200,4 & 41,2\\
          \hline
     \end{tabular}
\end{table}\\
Теперь изменяя ёмкость в диапазоне $0,00 - 0,09$ мкФ проведем измерения периодов свободных колебаний $T = 2 \pi \sqrt{LC}$.
\newpage
\begin{table}[htbp]
\centering
     \begin{tabular}{|c|c|c|c|c|}
     \hline
     C, мкФ & T\textsubscript{эксп}, мс & $\sigma_T$\textsubscript{эксп}, мс & T\textsubscript{теор}, мс & $\sigma_T$\textsubscript{теор}, мс \\ \hline
     0.000  & 0.072 & 0.004 & 0.000 & 0.000 \\
     0.001  & 0.092 & 0.006 & 0.087 & 0.005 \\
     0.002  & 0.118 & 0.008 & 0.122 & 0.008 \\
     0.003  & 0.135 & 0.009 & 0.149 & 0.010 \\
     0.004  & 0.151 & 0.010 & 0.172 & 0.012 \\
     0.005  & 0.164 & 0.011 & 0.193 & 0.013 \\
     0.006  & 0.175 & 0.012 & 0.211 & 0.015 \\
     0.007  & 0.187 & 0.013 & 0.228 & 0.016 \\
     0.008  & 0.198 & 0.014 & 0.244 & 0.017 \\
     0.009  & 0.209 & 0.015 & 0.259 & 0.018 \\
     \hline
     \end{tabular}
\end{table}
Зависимость периода свободных колебаний от емкости.
\begin{figure}[htbp]
    \centering
    \includegraphics[width=0.7\linewidth]{Tc.png}
    \label{fig:placeholder}
\end{figure}\\
Полученные данные соответствуют теоретической зависимости $T(C)$
\subsection{Измерение критического сопротивления и декремента затухания}
Ёмкость, при которой частота собственных колебаний контура будет равна $\nu_0 = 6.5$ кГц.
\begin{align}
     C = \frac{1}{4 \pi^2 \nu_0^2 L} \approx 5\ \text{нФ}
\end{align}
И для значений $L$ и $C$ рассчитаем $R_{\text{кр}}$
\begin{align}
     R_{\text{кр}} = 2\pi\sqrt{\dfrac{L}{C}} \approx 8164.97\ \text{Ом}
\end{align}
Для этих значений $L$ и $C$ рассчитаем декремент затухания для каждого сопротивления из интервала $(0,1-0,3)R_{crit}$. Из этих данных по формуле находим $R_{\text{кр}}$
\begin{align}
     R_{\text{кр}} = R_{\Sigma} \sqrt{\left[\dfrac{2\pi}{\theta}\right]^2 + 1}
\end{align}
     \begin{table}[htbp]
     \centering
          \begin{tabular}{|c|c|c|}
          \hline
          Параметр & Значение & Единица измерения \\ \hline
          $\nu_0$ & 6.5 & кГц \\
          $L$ & 100 & мГн \\
          $C^*$ & 0.006 & мкФ \\
          $R_{kr}$ & 8164.97 & Ом \\
          $R \approx 0.05R_{kr}$ & 408.24 & Ом \\ \hline
          \end{tabular}
     \end{table}
Посчитаем логарифмический декремент затуханий и запишем его в таблицу:
\begin{table}[htbp]
\centering
     \begin{tabular}{|c|c|c|c|c|}
     \hline
     $n$ & $R = n \times R_{kr}$, Ом & $U_1$, В & $U_2$, В & $\Theta$ \\
     \hline
     0.05 & 408.25 & 7.0 & 6.7 & 0.469 \\
     0.09 & 734.85 & 7.0 & 5.1 & 0.364 \\
     0.13 & 1061.45 & 6.5 & 4.3 & 0.349 \\
     0.17 & 1388.04 & 6.0 & 3.5 & 0.327 \\
     0.21 & 1714.64 & 5.3 & 3.0 & 0.298 \\
     0.25 & 2041.24 & 5.3 & 2.5 & 0.280 \\
     \hline
     \end{tabular}
\end{table}
$R_{\text{кр эксп}}=8051.32$\ Ом.
% \begin{figure}[H]
%     \centering
%     \includegraphics[width=0.3\linewidth]{fig.png}
% \end{figure}
\subsection{Свободные колебания на фазовой плоскости}
Рассмотрим свободные колебания на фазовой плоскости, для этого подключим место соединения
катушки индуктивности и магазина сопротивлений к выходу $X$ и включим на осциллографе канал $X-Y$.
\begin{table}[H]
\centering
     \begin{tabular}{|c|c|c|}
     \hline
     Параметр & Значение & Единица измерения \\ \hline
     $R$ & 408.25 & Ом \\
     $\nu_0$ & 100 & кГц \\
     $L$ & 100 & мГн \\
     $C^*$ & 0.006 & мкФ \\
     $R_{kr}$ & 8164.97 & Ом \\
     \hline
     \end{tabular}
\end{table}
Посчитаем логарифмический декремент по фазовой плоскости:
\begin{table}[H]
\centering
     \begin{tabular}{|c|c|c|c|}
     \hline
     $n$ & $R = n \times R_{kr}$, Ом & $U_1/U_2$ & $\Theta$ \\
     \hline
     0.05 & 408.25 & 1.27 & 0.480\\
     0.09 & 734.85 & 1.66 & 0.367\\
     0.13 & 1061.45 & 2.35 & 0.357\\
     0.17 & 1388.04 & 2.88 & 0.325\\
     0.21 & 1714.64 & 3.75 & 0.310\\
     0.25 & 2041.24 & 4.33 & 0.280\\
     \hline
     \end{tabular}
\label{tab:voltage_ratio}
\end{table}
График колебаний на фазовой плоскости.
\begin{figure}[H]
     \centering
     \includegraphics[width=0.4\linewidth]{phase.jpg}
\end{figure}
\subsection{Добротность свободных колебаний в контуре}
Найдем ее для $R_{max} = 3$ кОм и для $R_{min} = 1,8$ кОм из графика и фазовой диаграммы.
\begin{table}[htbp]
     \begin{center}
          \begin{tabular}{|c|c|c|c|c|c|c|c|}
          \hline
          \multirow{2}{*}{} & \multirow{2}{*}{$L$, мГн} & \multicolumn{3}{c|}{$R_{\text{кр}}$, кОм}                         & \multicolumn{3}{c|}{Q}                 \\ \cline{3-8} 
                         &                                  & Теор.                 & Подбор              & Граф.          & Теор. & Граф.         & Спираль        \\ \hline
          $R_{min}$         & \multirow{2}{*}{$100$}   & \multirow{2}{*}{8,2} & \multirow{2}{*}{8} & $8.1 \pm 0,1$ & 0.7   & $0.71 \pm 0,01$ & $0.68 \pm 0,01$  \\ \cline{1-1} \cline{5-8} 
          $R_{max}$         &                                  &                       &                     & $8,0 \pm 0,1$ & 1,3   & $ 1,23\pm 0,01$ & $1,12 \pm 0,01$ \\ \hline
          \end{tabular}
     \end{center}
\end{table}
\section{Вывод}
В ходе лабораторной работы были исследованы свободные и вынужденные колебания в RLC-контуре.
\end{document}